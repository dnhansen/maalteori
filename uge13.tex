\newcommand{\doctitle}{Uge 13}
\newcommand{\docauthor}{Danny Nygård Hansen}

\documentclass[a4paper, 11pt, article, danish, oneside]{memoir}
\usepackage[utf8]{inputenc}
\usepackage[T1]{fontenc}
\usepackage[danish]{babel}
\usepackage[autostyle, danish=guillemets]{csquotes}

\usepackage[final]{microtype}
\frenchspacing
\raggedbottom

\usepackage{mathtools}
\usepackage{amssymb}
\usepackage[largesmallcaps]{kpfonts}
\linespread{1.06}
\DeclareMathAlphabet\mathfrak{U}{euf}{m}{n}
\SetMathAlphabet\mathfrak{bold}{U}{euf}{b}{n}
\usepackage{inconsolata}

\usepackage{hyperref}
\hypersetup{%
	pdftitle=\doctitle,
	pdfauthor={\docauthor},
    hidelinks,
}

\usepackage{enumitem}
\setenumerate[0]{label=\normalfont(\arabic*)}
\setlist{
	listparindent=\parindent,
	parsep=0pt,
}
\usepackage{array}

\title{\doctitle}
\author{\docauthor}

\newcommand{\overbar}[3]{\mkern #1mu\overline{\mkern-#1mu#3\mkern-#2mu}\mkern #2mu}
\newcommand{\naturals}{\mathbb{N}}
\newcommand{\ints}{\mathbb{Z}}
\newcommand{\rationals}{\mathbb{Q}}
\newcommand{\reals}{\mathbb{R}}
\newcommand{\extreals}{\overbar{1.5}{1.5}{\reals}}
\newcommand{\complex}{\mathbb{C}}


\usepackage{pgffor}

\newcommand{\rvar}[1]{\mathsf{#1}}

\foreach \x in {A,...,Z}{%
    \expandafter\xdef\csname cal\x\endcsname{\noexpand\mathcal{\x}}
    \expandafter\xdef\csname frak\x\endcsname{\noexpand\mathfrak{\x}}
    \expandafter\xdef\csname rand\x\endcsname{\noexpand\rvar{\x}}
}


\usepackage{etoolbox}
\newcommand{\blank}{\mathrel{\;\cdot\;}}
\newcommand{\blankifempty}[1]{\ifstrempty{#1}{\blank}{#1}}
\DeclarePairedDelimiter{\auxdelimlvert}{\lvert}{\rvert}
\DeclarePairedDelimiter{\auxdelimlVert}{\lVert}{\rVert}
\DeclarePairedDelimiterX{\auxdelimanglescomma}[2]{\langle}{\rangle}{#1,#2}
\newcommand{\abs}[2][]{\auxdelimlvert[#1]{\blankifempty{#2}}}
\newcommand{\norm}[2][]{\auxdelimlVert[#1]{\blankifempty{#2}}}
\newcommand{\inner}[2]{\auxdelimanglescomma{\blankifempty{#1}}{\blankifempty{#2}}}


\DeclarePairedDelimiter{\auxdelimparen}{(}{)}
\DeclarePairedDelimiterX{\auxdelimparencomma}[2]{(}{)}{#1,#2}
\DeclarePairedDelimiter{\auxdelimbracket}{[}{]}
\DeclarePairedDelimiterX{\auxdelimbracketcomma}[2]{[}{]}{#1,#2}
\newcommand{\powerset}[2][]{\calP\auxdelimparen[#1]{#2}}
\newcommand{\borel}[2][]{\calB\auxdelimparen[#1]{#2}}
\newcommand{\meas}[2][]{\calM\auxdelimparen[#1]{#2}}
\newcommand{\measC}[2][]{\calM_\complex\auxdelimparen[#1]{#2}}
\newcommand{\measpos}[2][]{\meas[#1]{#2}^+}
\newcommand{\measbound}[2][]{\calM_b\auxdelimparen[#1]{#2}}
\newcommand{\measboundpos}[2][]{\measbound[#1]{#2}^+}


\newcommand{\extmeas}[2][]{\overbar{4.5}{0.5}{\calM}\auxdelimparen[#1]{#2}}
\newcommand{\extmeaspos}[2][]{\extmeas[#1]{#2}^+}
\newcommand{\simplemeas}[2][]{\calS\!\calM\auxdelimparen[#1]{#2}}
\newcommand{\simplemeaspos}[2][]{\simplemeas[#1]{#2}^+}
\newcommand{\sigmaalg}[2][]{\sigma\auxdelimparen[#1]{#2}}
\newcommand{\deltasys}[2][]{\delta\auxdelimparen[#1]{#2}}

\newcommand{\expval}[2][]{\mathbb{E}\auxdelimbracket[#1]{#2}}
\newcommand{\var}[2][]{\operatorname{Var}\auxdelimbracket[#1]{#2}}
\newcommand{\cov}[3][]{\operatorname{Cov}\auxdelimbracketcomma[#1]{#2}{#3}}


\renewcommand{\Re}{\operatorname{Re}}
\renewcommand{\Im}{\operatorname{Im}}
\newcommand{\conj}[1]{\overline{#1}}
\newcommand{\dif}{\mathop{}\!\mathrm{d}}
\DeclareMathOperator{\id}{id}
\newcommand{\indicator}[1]{\mathbf{1}_{#1}}

% Lattice operations
\newcommand{\meet}{\land}
\newcommand{\join}{\lor}

\DeclareMathOperator*{\smallbigvee}{\textstyle\bigvee}
\DeclareMathOperator*{\bigjoin}{\mathchoice
    {\smallbigvee}%
    {\bigvee}%
    {\bigvee}%
    {\bigvee}%
}
\DeclareMathOperator*{\smallbigwedge}{\textstyle\bigwedge}
\DeclareMathOperator*{\bigmeet}{\mathchoice
    {\smallbigwedge}%
    {\bigwedge}%
    {\bigwedge}%
    {\bigwedge}%
}



\newcommand*\union\cup
\newcommand*\intersect\cap

\DeclareMathOperator*{\smallbigcup}{\textstyle\bigcup}
\DeclareMathOperator*{\bigunion}{\mathchoice
    {\smallbigcup}%
    {\bigcup}%
    {\bigcup}%
    {\bigcup}%
}
\DeclareMathOperator*{\smallbigcap}{\textstyle\bigcap}
\DeclareMathOperator*{\bigintersect}{\mathchoice
    {\smallbigcap}%
    {\bigcap}%
    {\bigcap}%
    {\bigcap}%
}


\DeclarePairedDelimiterX{\set}[2]{\lbrace}{\rbrace}{#1\;\delimsize\vert\;#2}

\newcommand{\defeq}{\coloneqq}
\newcommand{\eqdef}{\eqqcolon}
\renewcommand{\phi}{\varphi}
\newcommand{\iu}{\mathrm{i}\mkern1mu}
\DeclareMathOperator{\e}{\mathrm{e}}

\newcommand{\ball}[3][]{%
    \ifstrempty{#1}%
        {%
            b\auxdelimparencomma{#2}{#3}%
        }{%
            b_{#1}\auxdelimparencomma{#2}{#3}%
        }%
}

\newcommand{\converges}[1]{\xrightarrow[#1]{}}
\DeclareMathOperator{\supp}{supp}
\let\oldvec\vec
\renewcommand{\vec}[1]{\underline{#1}}
\newcommand{\Tr}[1][]{%
    \ifstrempty{#1}%
        {%
            \operatorname{Tr}%
        }{%
            \operatorname{Tr}_{#1}%
        }%
}


\usepackage{listofitems}
\setsepchar{,}

\makeatletter
\newcommand{\mat@dims}[1]{%
    \readlist*\@dims{#1}%
    \ifnum \@dimslen=1
        \def\@dimsout{\@dims[1]}%
    \else
        \def\@dimsout{\@dims[1], \@dims[2]}%
    \fi
    \@dimsout
}


\newcommand{\matgroup}[3]{\mathrm{#1}_{#2}(#3)}
\newcommand{\matGL}[2]{\matgroup{GL}{#1}{#2}}
\newcommand{\trans}{^{\top}}
\newcommand{\mat}[2]{M_{\mat@dims{#1}}(#2)}

\makeatother

\DeclareMathOperator{\Span}{span}
\DeclareMathOperator{\clSpan}{\overbar{0.5}{1.5}{span}}

\newcommand\inv{^{-1}}
\newcommand{\preim}[2][]{^{-1}\auxdelimparen[#1]{#2}}

\newcommand{\dsupp}[2][]{\mathrm{Sp}_d\auxdelimparen[#1]{#2}}

\usepackage[amsmath,thmmarks,hyperref]{ntheorem}
\usepackage{bbding}

\newcommand{\theorembullet}{{\footnotesize\textbullet}}
\newcommand{\pencilsymbol}{\raisebox{-2pt}{\normalfont\PencilLeft}}
\makeatletter
\newtheoremstyle{changedotcustomnumber}%
    {}%
    {\item[\hskip\labelsep \theorem@headerfont ##3~~\theorembullet~~##1\theorem@separator]}
\newtheoremstyle{changedotbreakcustomnumber}%
    {}%
    {\item[\rlap{\vbox{\hbox{\hskip\labelsep \theorem@headerfont
            ##3~~\theorembullet~~##1\theorem@separator}\hbox{\strut}}}]}
\makeatother

\theorembodyfont{\normalfont}
\theoremseparator{~~}
\theoremsymbol{\ensuremath{\blacksquare}}
\theoremstyle{changedotcustomnumber}
\newtheorem{opgave}{\pencilsymbol}
\theoremstyle{changedotbreakcustomnumber}
\newtheorem{opgavebreak}{\pencilsymbol}

\newlist{solutionsec}{enumerate}{1}
\setlist[solutionsec]{leftmargin=0pt, parsep=0pt, listparindent=\parindent, label=(\alph*), labelsep=0pt, labelwidth=20pt, itemindent=20pt, align=left, itemsep=.5\baselineskip}


\begin{document}

\maketitle

% • 11.4, 11.5.
% • 12.2, 12.3, 12.4.
% Hvis man trænger til en ekstra udfordring, kan
% man regne Opgave A nedenfor.

\begin{opgavebreak}[11.4]
\begin{solutionsec}
    \item Bemærk at for $t \geq 0$ er
    %
    \begin{equation*}
        (\lambda_2 \circ \eta\inv)((-\infty,t])
            = \pi t^2
            = (f\lambda)((-\infty,t]),
    \end{equation*}
    %
    idet vi benytter at arealet af en disk med radius $t$ er $\pi t^2$.

    \item Bemærk at
    %
    \begin{align*}
        \int_{\reals^2} \e^{-x^2-y^2} \,\lambda_2(\dif x, \dif y)
            &= \int_\reals \e^{-r^2} \, (\lambda_2 \circ \eta\inv)(\dif r)
             = \int_\reals \e^{-r^2} \, (f\lambda)(\dif r) \\
            &= \int_\reals f(r) \e^{-r^2} \,\lambda(\dif r)
             = \pi.
    \end{align*}

    \item Følger nærmest direkte af Tonellis sætning.
    
    \item Benyt transformationssætningen (og evt. translationsinvarians).

    \item Benyt transformationssætningen sammen med del (c), og udnyt at funktionen $x \mapsto \e^{-x^2}$ er lige.
\end{solutionsec}
\end{opgavebreak}


\begin{opgavebreak}[11.5]
\begin{solutionsec}
    \item Husk at funktionen $x \mapsto \e^{-x^2/2}$ er et element i $\calL^1(\lambda)$.

    \item Funktionen $x \mapsto x \e^{-x^2/2}$ er også et element i $\calL^1(\lambda)$.

    \item Omskriv til Riemannintegraler og benyt partiel integration. (Første del af opgaven bruger ikke de foregående delopgaver.)

    \item At differentialligningen $y'(t) + ty(t) = 0$ har en entydig løsning $y$ der opfylder $y(0) = \sqrt{2\pi}$ bevises i kurset \emph{Differentialligninger}, men vi kan give et elementært bevis for denne påstand: Lad $y$ være en løsning til differentialligningen, og sæt $z(t) = \e^{t^2/2}y(t)$. Det er da let at vise at $z'(t) = 0$ (idet vi udnytter at $y$ er en løsning til differentialligningen), så $z \equiv C$ for et $C \in \reals$, og det følger at $y(t) = C \e^{-t^2/2}$. Vi ser desuden at $y(0) = C$, så vi må have $C = \sqrt{2\pi}$.
    
    Vi bemærker at resultatet af denne opgave benyttes i Eksempel~12.1.5 til at beregne den Fouriertransformerede af tætheden for normalfordelingen.
\end{solutionsec}
\end{opgavebreak}


\begin{opgave}[12.2]
    Ved Sætning~12.2.3(ii) er $f * g$ et element i $\calL^1_\complex(\lambda)$. Benyt Fubinis sætning (idet Tonellis sætning, som sædvanligt, viser at dette er tilladt).
\end{opgave}


\begin{opgave}[12.3]
    Dette er en direkte konsekvens af Inversionssætningen og Opgave~5.9(c).
\end{opgave}


\begin{opgavebreak}[12.4]
\begin{solutionsec}
    \item Benyt f.eks. Opgave~8.2(c) og bemærk at imaginærdelen er en ulige funktion.

    \item Sammenlign f.eks. med funktionen
    %
    \begin{equation*}
        t
            \mapsto \sqrt{\frac{2}{\pi}} \frac{\sigma}{t^2}
    \end{equation*}
    %
    for $\abs{t} \geq 1$. Denne er punktvist større end $\widehat{H}_\sigma$ og er som bekendt integrabel.

    \item Benyt Inversionssætningen og at imaginærdelen er en ulige funktion.
\end{solutionsec}
\end{opgavebreak}

\end{document}