\newcommand{\doctitle}{Uge 7}
\newcommand{\docauthor}{Danny Nygård Hansen}

\documentclass[a4paper, 11pt, article, danish, oneside]{memoir}
\usepackage[utf8]{inputenc}
\usepackage[T1]{fontenc}
\usepackage[danish]{babel}
\usepackage[autostyle, danish=guillemets]{csquotes}

\usepackage[final]{microtype}
\frenchspacing
\raggedbottom

\usepackage{mathtools}
\usepackage{amssymb}
\usepackage[largesmallcaps]{kpfonts}
\linespread{1.06}
\DeclareMathAlphabet\mathfrak{U}{euf}{m}{n}
\SetMathAlphabet\mathfrak{bold}{U}{euf}{b}{n}
\usepackage{inconsolata}

\usepackage{hyperref}
\hypersetup{%
	pdftitle=\doctitle,
	pdfauthor={\docauthor},
    hidelinks,
}

\usepackage{enumitem}
\setenumerate[0]{label=\normalfont(\arabic*)}
\setlist{
	listparindent=\parindent,
	parsep=0pt,
}
\usepackage{array}

\title{\doctitle}
\author{\docauthor}

\newcommand{\overbar}[3]{\mkern #1mu\overline{\mkern-#1mu#3\mkern-#2mu}\mkern #2mu}
\newcommand{\naturals}{\mathbb{N}}
\newcommand{\ints}{\mathbb{Z}}
\newcommand{\rationals}{\mathbb{Q}}
\newcommand{\reals}{\mathbb{R}}
\newcommand{\extreals}{\overbar{1.5}{1.5}{\reals}}
\newcommand{\complex}{\mathbb{C}}


\usepackage{pgffor}

\newcommand{\rvar}[1]{\mathsf{#1}}

\foreach \x in {A,...,Z}{%
    \expandafter\xdef\csname cal\x\endcsname{\noexpand\mathcal{\x}}
    \expandafter\xdef\csname frak\x\endcsname{\noexpand\mathfrak{\x}}
    \expandafter\xdef\csname rand\x\endcsname{\noexpand\rvar{\x}}
}


\usepackage{etoolbox}
\newcommand{\blank}{\mathrel{\;\cdot\;}}
\newcommand{\blankifempty}[1]{\ifstrempty{#1}{\blank}{#1}}
\DeclarePairedDelimiter{\auxdelimlvert}{\lvert}{\rvert}
\DeclarePairedDelimiter{\auxdelimlVert}{\lVert}{\rVert}
\DeclarePairedDelimiterX{\auxdelimanglescomma}[2]{\langle}{\rangle}{#1,#2}
\newcommand{\abs}[2][]{\auxdelimlvert[#1]{\blankifempty{#2}}}
\newcommand{\norm}[1]{\auxdelimlVert{\blankifempty{#1}}}
\newcommand{\inner}[2]{\auxdelimanglescomma{\blankifempty{#1}}{\blankifempty{#2}}}


\DeclarePairedDelimiter{\auxdelimparen}{(}{)}
\DeclarePairedDelimiterX{\auxdelimparencomma}[2]{(}{)}{#1,#2}
\DeclarePairedDelimiter{\auxdelimbracket}{[}{]}
\DeclarePairedDelimiterX{\auxdelimbracketcomma}[2]{[}{]}{#1,#2}
\newcommand{\powerset}[2][]{\calP\auxdelimparen[#1]{#2}}
\newcommand{\borel}[2][]{\calB\auxdelimparen[#1]{#2}}
\newcommand{\meas}[2][]{\calM\auxdelimparen[#1]{#2}}
\newcommand{\measC}[2][]{\calM_\complex\auxdelimparen[#1]{#2}}
\newcommand{\measpos}[2][]{\meas[#1]{#2}^+}
\newcommand{\measbound}[2][]{\calM_b\auxdelimparen[#1]{#2}}
\newcommand{\measboundpos}[2][]{\measbound[#1]{#2}^+}


\newcommand{\extmeas}[2][]{\overbar{4.5}{0.5}{\calM}\auxdelimparen[#1]{#2}}
\newcommand{\extmeaspos}[2][]{\extmeas[#1]{#2}^+}
\newcommand{\simplemeas}[2][]{\calS\!\calM\auxdelimparen[#1]{#2}}
\newcommand{\simplemeaspos}[2][]{\simplemeas[#1]{#2}^+}
\newcommand{\sigmaalg}[2][]{\sigma\auxdelimparen[#1]{#2}}
\newcommand{\deltasys}[2][]{\delta\auxdelimparen[#1]{#2}}

\newcommand{\expval}[2][]{\mathbb{E}\auxdelimbracket[#1]{#2}}
\newcommand{\var}[2][]{\operatorname{Var}\auxdelimbracket[#1]{#2}}
\newcommand{\cov}[3][]{\operatorname{Cov}\auxdelimbracketcomma[#1]{#2}{#3}}


\renewcommand{\Re}{\operatorname{Re}}
\renewcommand{\Im}{\operatorname{Im}}
\newcommand{\conj}[1]{\overline{#1}}
\newcommand{\dif}{\mathop{}\!\mathrm{d}}
\DeclareMathOperator{\id}{id}
\newcommand{\indicator}[1]{\mathbf{1}_{#1}}

% Lattice operations
\newcommand{\meet}{\land}
\newcommand{\join}{\lor}

\DeclareMathOperator*{\smallbigvee}{\textstyle\bigvee}
\DeclareMathOperator*{\bigjoin}{\mathchoice
    {\smallbigvee}%
    {\bigvee}%
    {\bigvee}%
    {\bigvee}%
}
\DeclareMathOperator*{\smallbigwedge}{\textstyle\bigwedge}
\DeclareMathOperator*{\bigmeet}{\mathchoice
    {\smallbigwedge}%
    {\bigwedge}%
    {\bigwedge}%
    {\bigwedge}%
}



\newcommand*\union\cup
\newcommand*\intersect\cap

\DeclareMathOperator*{\smallbigcup}{\textstyle\bigcup}
\DeclareMathOperator*{\bigunion}{\mathchoice
    {\smallbigcup}%
    {\bigcup}%
    {\bigcup}%
    {\bigcup}%
}
\DeclareMathOperator*{\smallbigcap}{\textstyle\bigcap}
\DeclareMathOperator*{\bigintersect}{\mathchoice
    {\smallbigcap}%
    {\bigcap}%
    {\bigcap}%
    {\bigcap}%
}


\DeclarePairedDelimiterX{\set}[2]{\lbrace}{\rbrace}{#1\;\delimsize\vert\;#2}

\newcommand{\defeq}{\coloneqq}
\newcommand{\eqdef}{\eqqcolon}
\renewcommand{\phi}{\varphi}
\newcommand{\iu}{\mathrm{i}\mkern1mu}
\DeclareMathOperator{\e}{\mathrm{e}}

\newcommand{\ball}[3][]{%
    \ifstrempty{#1}%
        {%
            b\auxdelimparencomma{#2}{#3}%
        }{%
            b_{#1}\auxdelimparencomma{#2}{#3}%
        }%
}

\newcommand{\converges}[1]{\xrightarrow[#1]{}}
\DeclareMathOperator{\supp}{supp}
\let\oldvec\vec
\renewcommand{\vec}[1]{\underline{#1}}
\newcommand{\Tr}[1][]{%
    \ifstrempty{#1}%
        {%
            \operatorname{Tr}%
        }{%
            \operatorname{Tr}_{#1}%
        }%
}


\usepackage{listofitems}
\setsepchar{,}

\makeatletter
\newcommand{\mat@dims}[1]{%
    \readlist*\@dims{#1}%
    \ifnum \@dimslen=1
        \def\@dimsout{\@dims[1]}%
    \else
        \def\@dimsout{\@dims[1], \@dims[2]}%
    \fi
    \@dimsout
}


\newcommand{\matgroup}[3]{\mathrm{#1}_{#2}(#3)}
\newcommand{\matGL}[2]{\matgroup{GL}{#1}{#2}}
\newcommand{\trans}{^{\top}}
\newcommand{\mat}[2]{M_{\mat@dims{#1}}(#2)}

\makeatother

\DeclareMathOperator{\Span}{span}
\DeclareMathOperator{\clSpan}{\overbar{0.5}{1.5}{span}}

\newcommand\inv{^{\langle-1\rangle}}
\newcommand{\preim}[2][]{^{-1}\auxdelimparen[#1]{#2}}

\newcommand{\dsupp}[2][]{\mathrm{Sp}_d\auxdelimparen[#1]{#2}}

\usepackage[amsmath,thmmarks,hyperref]{ntheorem}
\usepackage{bbding}

\newcommand{\theorembullet}{{\footnotesize\textbullet}}
\newcommand{\pencilsymbol}{\raisebox{-2pt}{\normalfont\PencilLeft}}
\makeatletter
\newtheoremstyle{changedotcustomnumber}%
    {}%
    {\item[\hskip\labelsep \theorem@headerfont ##3~~\theorembullet~~##1\theorem@separator]}
\newtheoremstyle{changedotbreakcustomnumber}%
    {}%
    {\item[\rlap{\vbox{\hbox{\hskip\labelsep \theorem@headerfont
            ##3~~\theorembullet~~##1\theorem@separator}\hbox{\strut}}}]}
\makeatother

\theorembodyfont{\normalfont}
\theoremseparator{~~}
\theoremsymbol{\ensuremath{\blacksquare}}
\theoremstyle{changedotcustomnumber}
\newtheorem{opgave}{\pencilsymbol}
\theoremstyle{changedotbreakcustomnumber}
\newtheorem{opgavebreak}{\pencilsymbol}

\newlist{solutionsec}{enumerate}{1}
\setlist[solutionsec]{leftmargin=0pt, parsep=0pt, listparindent=\parindent, label=(\alph*), labelsep=0pt, labelwidth=20pt, itemindent=20pt, align=left, itemsep=.5\baselineskip}


\begin{document}

\maketitle


\begin{opgavebreak}[2.1]
\begin{solutionsec}
    \item Vis betingelserne for fastholdt $B$ (vi har endda $\calU_\calJ = \bigintersect_{B \in \calJ} \calU_{\{B\}}$). For $A \in \calU_\calJ$ har vi
    %
    \begin{equation*}
        \mu((A_1 \setminus A_2) \intersect B)
            = \mu((A_1 \intersect B) \setminus A_2)
            = \mu(A_1 \intersect B) - \mu(A_2 \intersect B)
            = \mu(A_1 \setminus A_2) \mu(B).
    \end{equation*}
    %
    Den sidste betingelse følger ved kontinuitet af $\mu$.

    \item Betragt $A_1 \intersect A_2$.
\end{solutionsec}
\end{opgavebreak}


\begin{opgave}[2.2]
    Pointen er at der ikke er nok målelige mængder i $\calE$ til at skelne mellem $\mu$ og $\mu'$, hvilket der f.eks. er i Borelalgebraen på $\reals$.
\end{opgave}


\begin{opgave}[2.4]
    Sæt $\nu(B) = \mu(-B)$. Da stemmer $\nu$ og $\mu$ overens på mængderne $(-\infty,x]$. Benyt da Hovedsætning~2.2.2.
\end{opgave}


\begin{opgavebreak}[5.18]
\begin{solutionsec}
    \item Lad $n \in \naturals$ afparere $\epsilon = 1$, og bemærk at $\abs{f_n} - \abs{f} \leq \abs{f_n - f} \leq 1$. Vis da at $\int \abs{f} \dif\mu \leq \int \abs{f_n} \dif\mu + \mu(X)$.

    For den anden påstand, bemærk at $\abs{f} + 1$ er en integrabel majorant for en hale af følgen $(f_n)$, og anvend domineret konvergens. Alternativt bemærk at
    %
    \begin{equation*}
        \abs[\bigg]{ \int_X f_n \dif\mu - \int_X f \dif\mu }
            \leq \int_X \abs{ f_n - f } \dif\mu
            \leq \mu(X) \sup_{x \in X} \abs{f_n(x) - f(x)},
    \end{equation*}
    %
    og at højresiden går mod $0$.

    \item Betragt målrummet $(\reals,\borel{\reals},\mu)$. Hvis f.eks. $f_n = \sum_{j=1}^n (1/j) \indicator{(j-1,j]}$, så er alle $f_n$ integrable men ikke $f$. Hvis i stedet $f_n = (1/n) \indicator{[-n^2,n^2]}$, så er alle $f_n$ og $f = 0$ integrable, men $\int_\reals f_n \dif\lambda \to \infty$.
\end{solutionsec}
\end{opgavebreak}


\begin{opgavebreak}[5.26]
\begin{solutionsec}
    \item Benyt vinket samt resultatet fra Opgave~2.4.

    \item Bemærk at
    %
    \begin{equation*}
        \int f(x) \indicator{(0,\infty)}(x) \,\mu(\dif x)
            = \int f(-x) \indicator{(-\infty,0)}(-x) \,\mu(\dif x)
            = \int f(x) \indicator{(-\infty,0)}(x) \,\mu(\dif x).
    \end{equation*}
\end{solutionsec}
\end{opgavebreak}


\begin{opgavebreak}[6.3]
\begin{solutionsec}
    \item Benyt Tonelli og Fubini.

    \item Skriv $f(x) = x \e^{-x}$ og $g(y) = \e^{-y}$.
\end{solutionsec}
\end{opgavebreak}


\begin{opgavebreak}[6.4]
\begin{solutionsec}
    \item For at vise at $S$ er en Borelmængde, gå f.eks. frem som i Opgave~4.7(c) eller Opgave~4.8.

    \item Se Eksempel~6.4.3.

    \item Se Eksempel~6.4.3.
\end{solutionsec}
\end{opgavebreak}


\begin{opgave}[6.5]
    Som i Opgave~6.4.
\end{opgave}


\begin{opgavebreak}[6.6]
\begin{solutionsec}
    \item Som i Opgave~6.4 og 6.5.

    \item Bemærk at tællemålet på $\reals$ ikke er $\sigma$-endeligt.
\end{solutionsec}
\end{opgavebreak}


\begin{opgave}[6.7]
    Som i ovenstående opgaver.
\end{opgave}


\end{document}