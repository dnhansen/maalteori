\newcommand{\doctitle}{Uge 12}
\newcommand{\docauthor}{Danny Nygård Hansen}

\documentclass[a4paper, 11pt, article, danish, oneside]{memoir}
\usepackage[utf8]{inputenc}
\usepackage[T1]{fontenc}
\usepackage[danish]{babel}
\usepackage[autostyle, danish=guillemets]{csquotes}

\usepackage[final]{microtype}
\frenchspacing
\raggedbottom

\usepackage{mathtools}
\usepackage{amssymb}
\usepackage[largesmallcaps]{kpfonts}
\linespread{1.06}
\DeclareMathAlphabet\mathfrak{U}{euf}{m}{n}
\SetMathAlphabet\mathfrak{bold}{U}{euf}{b}{n}
\usepackage{inconsolata}

\usepackage{hyperref}
\hypersetup{%
	pdftitle=\doctitle,
	pdfauthor={\docauthor},
    hidelinks,
}

\usepackage{enumitem}
\setenumerate[0]{label=\normalfont(\arabic*)}
\setlist{
	listparindent=\parindent,
	parsep=0pt,
}
\usepackage{array}

\title{\doctitle}
\author{\docauthor}

\newcommand{\overbar}[3]{\mkern #1mu\overline{\mkern-#1mu#3\mkern-#2mu}\mkern #2mu}
\newcommand{\naturals}{\mathbb{N}}
\newcommand{\ints}{\mathbb{Z}}
\newcommand{\rationals}{\mathbb{Q}}
\newcommand{\reals}{\mathbb{R}}
\newcommand{\extreals}{\overbar{1.5}{1.5}{\reals}}
\newcommand{\complex}{\mathbb{C}}


\usepackage{pgffor}

\newcommand{\rvar}[1]{\mathsf{#1}}

\foreach \x in {A,...,Z}{%
    \expandafter\xdef\csname cal\x\endcsname{\noexpand\mathcal{\x}}
    \expandafter\xdef\csname frak\x\endcsname{\noexpand\mathfrak{\x}}
    \expandafter\xdef\csname rand\x\endcsname{\noexpand\rvar{\x}}
}


\usepackage{etoolbox}
\newcommand{\blank}{\mathrel{\;\cdot\;}}
\newcommand{\blankifempty}[1]{\ifstrempty{#1}{\blank}{#1}}
\DeclarePairedDelimiter{\auxdelimlvert}{\lvert}{\rvert}
\DeclarePairedDelimiter{\auxdelimlVert}{\lVert}{\rVert}
\DeclarePairedDelimiterX{\auxdelimanglescomma}[2]{\langle}{\rangle}{#1,#2}
\newcommand{\abs}[2][]{\auxdelimlvert[#1]{\blankifempty{#2}}}
\newcommand{\norm}[2][]{\auxdelimlVert[#1]{\blankifempty{#2}}}
\newcommand{\inner}[2]{\auxdelimanglescomma{\blankifempty{#1}}{\blankifempty{#2}}}


\DeclarePairedDelimiter{\auxdelimparen}{(}{)}
\DeclarePairedDelimiterX{\auxdelimparencomma}[2]{(}{)}{#1,#2}
\DeclarePairedDelimiter{\auxdelimbracket}{[}{]}
\DeclarePairedDelimiterX{\auxdelimbracketcomma}[2]{[}{]}{#1,#2}
\newcommand{\powerset}[2][]{\calP\auxdelimparen[#1]{#2}}
\newcommand{\borel}[2][]{\calB\auxdelimparen[#1]{#2}}
\newcommand{\meas}[2][]{\calM\auxdelimparen[#1]{#2}}
\newcommand{\measC}[2][]{\calM_\complex\auxdelimparen[#1]{#2}}
\newcommand{\measpos}[2][]{\meas[#1]{#2}^+}
\newcommand{\measbound}[2][]{\calM_b\auxdelimparen[#1]{#2}}
\newcommand{\measboundpos}[2][]{\measbound[#1]{#2}^+}


\newcommand{\extmeas}[2][]{\overbar{4.5}{0.5}{\calM}\auxdelimparen[#1]{#2}}
\newcommand{\extmeaspos}[2][]{\extmeas[#1]{#2}^+}
\newcommand{\simplemeas}[2][]{\calS\!\calM\auxdelimparen[#1]{#2}}
\newcommand{\simplemeaspos}[2][]{\simplemeas[#1]{#2}^+}
\newcommand{\sigmaalg}[2][]{\sigma\auxdelimparen[#1]{#2}}
\newcommand{\deltasys}[2][]{\delta\auxdelimparen[#1]{#2}}

\newcommand{\expval}[2][]{\mathbb{E}\auxdelimbracket[#1]{#2}}
\newcommand{\var}[2][]{\operatorname{Var}\auxdelimbracket[#1]{#2}}
\newcommand{\cov}[3][]{\operatorname{Cov}\auxdelimbracketcomma[#1]{#2}{#3}}


\renewcommand{\Re}{\operatorname{Re}}
\renewcommand{\Im}{\operatorname{Im}}
\newcommand{\conj}[1]{\overline{#1}}
\newcommand{\dif}{\mathop{}\!\mathrm{d}}
\DeclareMathOperator{\id}{id}
\newcommand{\indicator}[1]{\mathbf{1}_{#1}}

% Lattice operations
\newcommand{\meet}{\land}
\newcommand{\join}{\lor}

\DeclareMathOperator*{\smallbigvee}{\textstyle\bigvee}
\DeclareMathOperator*{\bigjoin}{\mathchoice
    {\smallbigvee}%
    {\bigvee}%
    {\bigvee}%
    {\bigvee}%
}
\DeclareMathOperator*{\smallbigwedge}{\textstyle\bigwedge}
\DeclareMathOperator*{\bigmeet}{\mathchoice
    {\smallbigwedge}%
    {\bigwedge}%
    {\bigwedge}%
    {\bigwedge}%
}



\newcommand*\union\cup
\newcommand*\intersect\cap

\DeclareMathOperator*{\smallbigcup}{\textstyle\bigcup}
\DeclareMathOperator*{\bigunion}{\mathchoice
    {\smallbigcup}%
    {\bigcup}%
    {\bigcup}%
    {\bigcup}%
}
\DeclareMathOperator*{\smallbigcap}{\textstyle\bigcap}
\DeclareMathOperator*{\bigintersect}{\mathchoice
    {\smallbigcap}%
    {\bigcap}%
    {\bigcap}%
    {\bigcap}%
}


\DeclarePairedDelimiterX{\set}[2]{\lbrace}{\rbrace}{#1\;\delimsize\vert\;#2}

\newcommand{\defeq}{\coloneqq}
\newcommand{\eqdef}{\eqqcolon}
\renewcommand{\phi}{\varphi}
\newcommand{\iu}{\mathrm{i}\mkern1mu}
\DeclareMathOperator{\e}{\mathrm{e}}

\newcommand{\ball}[3][]{%
    \ifstrempty{#1}%
        {%
            b\auxdelimparencomma{#2}{#3}%
        }{%
            b_{#1}\auxdelimparencomma{#2}{#3}%
        }%
}

\newcommand{\converges}[1]{\xrightarrow[#1]{}}
\DeclareMathOperator{\supp}{supp}
\let\oldvec\vec
\renewcommand{\vec}[1]{\underline{#1}}
\newcommand{\Tr}[1][]{%
    \ifstrempty{#1}%
        {%
            \operatorname{Tr}%
        }{%
            \operatorname{Tr}_{#1}%
        }%
}


\usepackage{listofitems}
\setsepchar{,}

\makeatletter
\newcommand{\mat@dims}[1]{%
    \readlist*\@dims{#1}%
    \ifnum \@dimslen=1
        \def\@dimsout{\@dims[1]}%
    \else
        \def\@dimsout{\@dims[1], \@dims[2]}%
    \fi
    \@dimsout
}


\newcommand{\matgroup}[3]{\mathrm{#1}_{#2}(#3)}
\newcommand{\matGL}[2]{\matgroup{GL}{#1}{#2}}
\newcommand{\trans}{^{\top}}
\newcommand{\mat}[2]{M_{\mat@dims{#1}}(#2)}

\makeatother

\DeclareMathOperator{\Span}{span}
\DeclareMathOperator{\clSpan}{\overbar{0.5}{1.5}{span}}

\newcommand\inv{^{-1}}
\newcommand{\preim}[2][]{^{-1}\auxdelimparen[#1]{#2}}

\newcommand{\dsupp}[2][]{\mathrm{Sp}_d\auxdelimparen[#1]{#2}}

\usepackage[amsmath,thmmarks,hyperref]{ntheorem}
\usepackage{bbding}

\newcommand{\theorembullet}{{\footnotesize\textbullet}}
\newcommand{\pencilsymbol}{\raisebox{-2pt}{\normalfont\PencilLeft}}
\makeatletter
\newtheoremstyle{changedotcustomnumber}%
    {}%
    {\item[\hskip\labelsep \theorem@headerfont ##3~~\theorembullet~~##1\theorem@separator]}
\newtheoremstyle{changedotbreakcustomnumber}%
    {}%
    {\item[\rlap{\vbox{\hbox{\hskip\labelsep \theorem@headerfont
            ##3~~\theorembullet~~##1\theorem@separator}\hbox{\strut}}}]}
\makeatother

\theorembodyfont{\normalfont}
\theoremseparator{~~}
\theoremsymbol{\ensuremath{\blacksquare}}
\theoremstyle{changedotcustomnumber}
\newtheorem{opgave}{\pencilsymbol}
\theoremstyle{changedotbreakcustomnumber}
\newtheorem{opgavebreak}{\pencilsymbol}

\newlist{solutionsec}{enumerate}{1}
\setlist[solutionsec]{leftmargin=0pt, parsep=0pt, listparindent=\parindent, label=(\alph*), labelsep=0pt, labelwidth=20pt, itemindent=20pt, align=left, itemsep=.5\baselineskip}


\begin{document}

\maketitle

% 9.13(a)-(c), 9.15.
% • Opgave 1 nedenfor.
% • 10.1
% , 10.3.

\begin{opgavebreak}[9.13]
\begin{solutionsec}
    \item Den eneste $\tau$-nulmængde er $\emptyset$.

    \item Bemærk at
    %
    \begin{equation*}
        \inner{f}{\indicator{\{x\}}}
            = \int_X f \indicator{\{x\}} \dif\tau
            = \int_X f(x) \indicator{\{x\}} \dif\tau
            = f(x) \tau(\{x\})
            = f(x).
    \end{equation*}

    \item I lyset af del (b) har vi
    %
    \begin{equation*}
        f
            = \sum_{x \in X} f(x) \indicator{\{x\}}
            = \sum_{x \in X} \inner{f}{\indicator{\{x\}}} \indicator{\{x\}},
    \end{equation*}
    %
    for enhver funktion $f \in L^2(\tau)$. Alternativt, hvis $f$ er ortogonal på enhver $\indicator{\{x\}}$, så må den være $0$ overalt. Altså opfylder mængden betingelserne (i) hhv. (iv) i Korollar~9.4.9.
\end{solutionsec}
\end{opgavebreak}


\begin{opgavebreak}[9.15]
\begin{solutionsec}
    \item Vis at hver betingelse medfører den næste, og at (v) medfører (i). For at vise implikationen (ii) $\Rightarrow$ (iii), vælg $\delta > 0$ så $\norm{x} \leq \delta$ medfører $\norm{T(x)} \leq 1$, og bemærk at
    %
    \begin{equation*}
		\norm{T(x)}
			= \frac{\norm{x}}{\delta} \norm[\bigg]{T \biggl( \delta \frac{x}{\norm{x}} \biggr) }
			\leq \delta\inv \norm{x}.
	\end{equation*}
    %
    For et mere geometrisk argument, lad først
    %
    \begin{equation*}
        \overline{b}(x,r)
            = \set[\big]{y \in V}{\norm{x-y} \leq r}
    \end{equation*}
    %
    betegne den \emph{lukkede} kugle med centrum i $x \in V$ og radius $r > 0$. Bemærk så at kontinuitet af $T$ i $0$ betyder at
    %
    \begin{equation*}
        T\bigl( \overline{b}(0,\delta) \bigr)
            \subseteq \overline{b}(0,1).
    \end{equation*}
    %
    Dette medfører at
    %
    \begin{equation*}
        T\bigl( \overline{b}(0,1) \bigr)
            = T\bigl( \tfrac{1}{\delta} \overline{b}(0,\delta) \bigr)
            = \tfrac{1}{\delta} T\bigl( \overline{b}(0,\delta) \bigr)
            \subseteq \tfrac{1}{\delta} \overline{b}(0,1)
            = \overline{b}(0,\tfrac{1}{\delta}),
    \end{equation*}
    %
    hvilket netop siger at $\norm{T(x)} \leq \tfrac{1}{\delta}$ når $\norm{x} \leq 1$.

    \item For trekantsuligheden, bemærk at
    %
    \begin{equation*}
        \norm{T(x) + S(x)}
            \leq \norm{T(x)} + \norm{S(x)}
            \leq \norm{T} + \norm{S}
    \end{equation*}
    %
    for $\norm{x} \leq 1$. For homogenitet, bemærk at
    %
    \begin{equation*}
        \norm{\alpha T(x)}
            = \abs{\alpha} \, \norm{T(x)}
            \leq \abs{\alpha} \, \norm{T}
    \end{equation*}
    %
    for samme $x$. Den omvendte ulighed følger ved at lave substitutionerne $\alpha \to \alpha\inv$ og $T \to \alpha T$. Uligheden følger ved at erstatte $x$ med $x/\norm{x}$.

    \item Følger direkte af uligheden vist i del (b).
\end{solutionsec}
\end{opgavebreak}


\begin{opgavebreak}[US1]
\begin{solutionsec}
    \item For alle $A \in \calE$ med $\mu(A) = 0$ er
    %
    \begin{equation*}
        \nu(A)
            = \int_A h \dif\mu
            = \int_X h \indicator{A} \dif\mu
            = 0,
    \end{equation*}
    %
    da $h \indicator{A} = 0$ $\mu$-n.o.

    \item Her er $\lambda \ll \tau$ (vi har endda $\mu \ll \tau$ for alle Borelmål $\mu$), men ikke omvendt.
\end{solutionsec}
\end{opgavebreak}


\newcommand{\Bin}[2]{\mathrm{Bin}(#1,#2)}

\begin{opgave}[10.1]
    Alle mål på $\naturals$ er absolut kontinuerte med hensyn til tællemålet. Bemærk dernæst at hvis $A \subseteq \naturals$, så er
    %
    \begin{align*}
        \Bin{n}{p}(A)
            &= \sum_{k=0}^n f(k) \delta_k(A)
             = \sum_{k=0}^n f(k) \tau_\naturals(A \intersect \{k\}) \\
            &= \sum_{k=0}^n f(k) \int_A \indicator{\{k\}} \dif\tau_\naturals
             = \int_A \sum_{k=0}^n f(k) \indicator{\{k\}} \dif\tau_\naturals \\
            &= \int_A f \dif\tau_\naturals.
    \end{align*}
\end{opgave}


\begin{opgavebreak}[10.3]
\begin{solutionsec}
    \item Dette er oplagt.

    \item Det følger af Sætning~10.1.4 at
    %
    \begin{equation*}
        \nu(A)
            = \int_A f \dif\mu
            = \int_A fg \dif\varpi.
    \end{equation*}
    %
    Såfremt tæthederne er entydigt bestemte (næsten overalt), har vi altså (jf. notationen fra Definition~10.1.2)
    %
    \begin{equation*}
        \frac{\dif\nu}{\dif\varpi}
            = \frac{\dif\nu}{\dif\mu} \frac{\dif\mu}{\dif\varpi},
    \end{equation*}
    %
    $\varpi$-n.o. (Bemærk desuden at bogstavet \enquote{$\varpi$} ikke er et omega, men derimod en variant af pi. I \LaTeX{} kan kommandoen \verb|\varpi| benyttes.)
\end{solutionsec}
\end{opgavebreak}


\end{document}