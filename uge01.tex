\newcommand{\doctitle}{Uge 1}
\newcommand{\docauthor}{Danny Nygård Hansen}

\documentclass[a4paper, 11pt, article, danish, oneside]{memoir}
\usepackage[utf8]{inputenc}
\usepackage[T1]{fontenc}
\usepackage[danish]{babel}
\usepackage[autostyle, danish=guillemets]{csquotes}

\usepackage[final]{microtype}
\frenchspacing
\raggedbottom

\usepackage{mathtools}
\usepackage{amssymb}
\usepackage{kpfonts}
\linespread{1.06}
\DeclareMathAlphabet\mathfrak{U}{euf}{m}{n}
\SetMathAlphabet\mathfrak{bold}{U}{euf}{b}{n}
\usepackage{inconsolata}

\usepackage{hyperref}
\hypersetup{%
	pdftitle=\doctitle,
	pdfauthor={\docauthor},
}

\usepackage{enumitem}
\setenumerate[0]{label=\normalfont(\arabic*)}
\setlist{
	listparindent=\parindent,
	parsep=0pt,
}
\usepackage{array}

\title{\doctitle}
\author{\docauthor}

\newcommand{\overbar}[3]{\mkern #1mu\overline{\mkern-#1mu#3\mkern-#2mu}\mkern #2mu}
\newcommand{\naturals}{\mathbb{N}}
\newcommand{\ints}{\mathbb{Z}}
\newcommand{\rationals}{\mathbb{Q}}
\newcommand{\reals}{\mathbb{R}}
\newcommand{\extreals}{\overbar{1.5}{1.5}{\reals}}
\newcommand{\complex}{\mathbb{C}}


\usepackage{pgffor}

\newcommand{\rvar}[1]{\mathsf{#1}}

\foreach \x in {A,...,Z}{%
    \expandafter\xdef\csname cal\x\endcsname{\noexpand\mathcal{\x}}
    \expandafter\xdef\csname frak\x\endcsname{\noexpand\mathfrak{\x}}
    \expandafter\xdef\csname rand\x\endcsname{\noexpand\rvar{\x}}
}


\usepackage{etoolbox}
\newcommand{\blank}{\mathrel{\;\cdot\;}}
\newcommand{\blankifempty}[1]{\ifstrempty{#1}{\blank}{#1}}
\DeclarePairedDelimiter{\auxdelimlvert}{\lvert}{\rvert}
\DeclarePairedDelimiter{\auxdelimlVert}{\lVert}{\rVert}
\DeclarePairedDelimiterX{\auxdelimanglescomma}[2]{\langle}{\rangle}{#1,#2}
\newcommand{\abs}[1]{\auxdelimlvert{\blankifempty{#1}}}
\newcommand{\norm}[1]{\auxdelimlVert{\blankifempty{#1}}}
\newcommand{\inner}[2]{\auxdelimanglescomma{\blankifempty{#1}}{\blankifempty{#2}}}


\DeclarePairedDelimiter{\auxdelimparen}{(}{)}
\DeclarePairedDelimiterX{\auxdelimparencomma}[2]{(}{)}{#1,#2}
\DeclarePairedDelimiter{\auxdelimbracket}{[}{]}
\DeclarePairedDelimiterX{\auxdelimbracketcomma}[2]{[}{]}{#1,#2}
\newcommand{\powerset}[2][]{\calP\auxdelimparen[#1]{#2}}
\newcommand{\borel}[2][]{\calB\auxdelimparen[#1]{#2}}
\newcommand{\meas}[2][]{\calM\auxdelimparen[#1]{#2}}
\newcommand{\measC}[2][]{\calM_\complex\auxdelimparen[#1]{#2}}
\newcommand{\measpos}[2][]{\meas[#1]{#2}^+}
\newcommand{\measbound}[2][]{\calM_b\auxdelimparen[#1]{#2}}
\newcommand{\measboundpos}[2][]{\measbound[#1]{#2}^+}


\newcommand{\extmeas}[2][]{\overbar{4.5}{0.5}{\calM}\auxdelimparen[#1]{#2}}
\newcommand{\extmeaspos}[2][]{\extmeas[#1]{#2}^+}
\newcommand{\simplemeas}[2][]{\calS\!\calM\auxdelimparen[#1]{#2}}
\newcommand{\simplemeaspos}[2][]{\simplemeas[#1]{#2}^+}
\newcommand{\sigmaalg}[2][]{\sigma\auxdelimparen[#1]{#2}}
\newcommand{\deltasys}[2][]{\delta\auxdelimparen[#1]{#2}}

\newcommand{\expval}[2][]{\mathbb{E}\auxdelimbracket[#1]{#2}}
\newcommand{\var}[2][]{\operatorname{Var}\auxdelimbracket[#1]{#2}}
\newcommand{\cov}[3][]{\operatorname{Cov}\auxdelimbracketcomma[#1]{#2}{#3}}


\renewcommand{\Re}{\operatorname{Re}}
\renewcommand{\Im}{\operatorname{Im}}
\newcommand{\conj}[1]{\overline{#1}}
\newcommand{\dif}{\mathop{}\!\mathrm{d}}
\DeclareMathOperator{\id}{id}
\newcommand{\indicator}[1]{\mathbf{1}_{#1}}

% Lattice operations
\newcommand{\meet}{\land}
\newcommand{\join}{\lor}

\DeclareMathOperator*{\smallbigvee}{\textstyle\bigvee}
\DeclareMathOperator*{\bigjoin}{\mathchoice
    {\smallbigvee}%
    {\bigvee}%
    {\bigvee}%
    {\bigvee}%
}
\DeclareMathOperator*{\smallbigwedge}{\textstyle\bigwedge}
\DeclareMathOperator*{\bigmeet}{\mathchoice
    {\smallbigwedge}%
    {\bigwedge}%
    {\bigwedge}%
    {\bigwedge}%
}



\newcommand*\union\cup
\newcommand*\intersect\cap

\DeclareMathOperator*{\smallbigcup}{\textstyle\bigcup}
\DeclareMathOperator*{\bigunion}{\mathchoice
    {\smallbigcup}%
    {\bigcup}%
    {\bigcup}%
    {\bigcup}%
}
\DeclareMathOperator*{\smallbigcap}{\textstyle\bigcap}
\DeclareMathOperator*{\bigintersect}{\mathchoice
    {\smallbigcap}%
    {\bigcap}%
    {\bigcap}%
    {\bigcap}%
}


\DeclarePairedDelimiterX{\set}[2]{\lbrace}{\rbrace}{#1\;\delimsize\vert\;#2}

\newcommand{\defeq}{\coloneqq}
\renewcommand{\phi}{\varphi}
\newcommand{\iu}{\mathrm{i}\mkern1mu}
\DeclareMathOperator{\e}{\mathrm{e}}

\newcommand{\ball}[3][]{%
    \ifstrempty{#1}%
        {%
            b\auxdelimparencomma{#2}{#3}%
        }{%
            b_{#1}\auxdelimparencomma{#2}{#3}%
        }%
}

\newcommand{\converges}[1]{\xrightarrow[#1]{}}
\DeclareMathOperator{\supp}{supp}
\let\oldvec\vec
\renewcommand{\vec}[1]{\underline{#1}}
\newcommand{\Tr}[1][]{%
    \ifstrempty{#1}%
        {%
            \operatorname{Tr}%
        }{%
            \operatorname{Tr}_{#1}%
        }%
}


\usepackage{listofitems}
\setsepchar{,}

\makeatletter
\newcommand{\mat@dims}[1]{%
    \readlist*\@dims{#1}%
    \ifnum \@dimslen=1
        \def\@dimsout{\@dims[1]}%
    \else
        \def\@dimsout{\@dims[1], \@dims[2]}%
    \fi
    \@dimsout
}


\newcommand{\matgroup}[3]{\mathrm{#1}_{#2}(#3)}
\newcommand{\matGL}[2]{\matgroup{GL}{#1}{#2}}
\newcommand{\trans}{^{\top}}
\newcommand{\mat}[2]{M_{\mat@dims{#1}}(#2)}

\makeatother

\DeclareMathOperator{\Span}{span}
\DeclareMathOperator{\clSpan}{\overbar{0.5}{1.5}{span}}

\newcommand\inv{^{\langle-1\rangle}}
\newcommand{\preim}[2][]{^{-1}\auxdelimparen[#1]{#2}}

\newcommand{\dsupp}[2][]{\mathrm{Sp}_d\auxdelimparen[#1]{#2}}

\usepackage[amsmath,thmmarks,hyperref]{ntheorem}
\usepackage{bbding}

\newcommand{\theorembullet}{{\footnotesize\textbullet}}
\newcommand{\pencilsymbol}{\raisebox{-2pt}{\normalfont\PencilLeft}}
\makeatletter
\newtheoremstyle{changedotcustomnumber}%
    {}%
    {\item[\hskip\labelsep \theorem@headerfont ##3~~\theorembullet~~##1\theorem@separator]}
\newtheoremstyle{changedotbreakcustomnumber}%
    {}%
    {\item[\rlap{\vbox{\hbox{\hskip\labelsep \theorem@headerfont
            ##3~~\theorembullet~~##1\theorem@separator}\hbox{\strut}}}]}
\makeatother

\theorembodyfont{\normalfont}
\theoremseparator{~~}
\theoremsymbol{\ensuremath{\blacksquare}}
\theoremstyle{changedotcustomnumber}
\newtheorem{opgave}{\pencilsymbol}
\theoremstyle{changedotbreakcustomnumber}
\newtheorem{opgavebreak}{\pencilsymbol}

\newlist{solutionsec}{enumerate}{1}
\setlist[solutionsec]{leftmargin=0pt, parsep=0pt, listparindent=\parindent, label=(\alph*), labelsep=0pt, labelwidth=20pt, itemindent=20pt, align=left, itemsep=.5\baselineskip}


\begin{document}

\maketitle

\begin{opgavebreak}[US1]
\begin{solutionsec}
    \item To mængder er (pr. definition) ens hvis de har samme elementer. For f.eks. at vise (A.9), altså at $(A \union B)^c = A^c \intersect B^c$, skal man vise at ethvert element i mængden til venstre for lighedstegnet også ligger i mængden til højre for lighedstegnet, og omvendt. Det kan nogle gange være nyttigt at vise inklusionerne \textquote{$\subseteq$} og \textquote{$\supseteq$} hver for sig. Det er også ofte nyttigt at tegne en figur der repræsenterer mængderne som geometriske figurer (f.eks. cirkler) og lade dette styre ens intuition.

    For at vise et udsagn som (A.17), altså at $A \subseteq B$ medfører $B = A \union (B \setminus A)$, skal man gøre det samme men undervejs benytte antagelsen $A \subseteq B$.

    Bemærk at (A.9) og (A.10) er specialtilfælde af (A.20).

    \item Her er $\calA_3$ stabilt over for $\union$ og $\intersect$, men ikke komplementærmængdedannelse. Derudover er f.eks. $(-\infty,1) = \bigunion_{n \in \naturals} (-\infty, 1-\frac{1}{n}]$, så $\calA_3$ er ikke stabilt over for forening af tælleligt mange mængder.

    Yderligere er $\calA_4$ ikke stabilt over for komplementærmængdedannelse.

    \item Disse udsagn vises med samme teknikker som i (a), idet man selvfølgelig benytter definitionen på urbillede. Hvis f.eks. $x \in f\preim{H^c}$, da er $f(x) \in H^c$. Altså er $f(x) \not\in H$, så $x \not\in f\preim{H}$, og dermed er $x \in f\preim{H}^c$.

    \item Der er selvfølgelig ikke en præcis definition på \textquote{meningsfuldt}, og strengt taget giver alle udsagnene mening. Men udsagnene
    %
    \begin{equation*}
        X \subseteq \calG,
        \quad
        \calG \subseteq X
        \quad \text{og} \quad
        \{1,3\} \intersect \calG = \emptyset
    \end{equation*}
    %
    er nok ikke udsagn som vi rent faktisk er interesserede i.
\end{solutionsec}
\end{opgavebreak}


\begin{opgavebreak}[1.1]
\begin{solutionsec}
    \item Den eneste svære egenskab at vise er nok trekantsuligheden. Her skal man benytte trekantsuligheden for absolutværdien på $\reals$, samt bemærke at
    %
    \begin{equation*}
        \max_{i=1,\ldots,d} \bigl( \abs{x_i - y_i} + \abs{y_i - z_i} \bigr)
            \leq \max_{i=1,\ldots,d} \abs{x_i - y_i} + \max_{i=1,\ldots,d} \abs{y_i - z_i}.
    \end{equation*}

    \item Her er $b_2(\underline{0},2)$ en cirkelskive (uden rand) med radius $2$, og $b_\infty(\underline{0},2)$ er et kvadrat (uden rand) med sidelængde $4$.

    \item Bemærk at hvis $a_1, \ldots, a_d \geq 0$, så er
    %
    \begin{equation*}
        \max\{a_1, \ldots, a_d\}
            = \bigl( \max\{a_1^2, \ldots, a_d^2\} \bigr)^{1/2}
            \leq \bigl( a_1^2 + \cdots + a_d^2 \bigr)^{1/2},
    \end{equation*}
    %
    og omvendt er
    %
    \begin{equation*}
        \bigl( a_1^2 + \cdots + a_d^2 \bigr)^{1/2}
            \leq \bigl( d \max\{a_1^2, \ldots, a_d^2\} \bigr)^{1/2}
            = \sqrt{d} \max\{a_1, \ldots, a_d\}.
    \end{equation*}

    \item Benyt (c).

    \item Benyt (d).
\end{solutionsec}
\end{opgavebreak}


\begin{opgavebreak}[1.2]
\begin{solutionsec}
    \item Her hjælper det at tegne en figur. For $\rationals$, lad $a,b \in \reals$ med $a < b$. Vælg et $n \in \naturals$ så $\frac{1}{n} < b-a$, og lad $m \in \ints$ være det mindste heltal med $m > an$. Så er $a < \frac{m}{n} < b$, hvor den sidste ulighed følger da $m-1 \leq an$, så
    %
    \begin{equation*}
        \frac{m}{n}
            = \frac{m-1}{n} + \frac{1}{n}
            < a + b - a
            = b.
    \end{equation*}
    %
    For $\reals \setminus \rationals$ kan man f.eks. gøre det samme, men erstatte $n$ med $\sqrt{2}n$. Man kan også, for $a < b$, finde et $q \in \rationals$ så
    %
    \begin{equation*}
        \frac{a}{\sqrt{2}}
            < q
            < \frac{b}{\sqrt{2}},
    \end{equation*}
    %
    hvilket medfører at
    %
    \begin{equation*}
        a
            < \sqrt{2}q
            < b,
    \end{equation*}
    %
    som ønsket.

    \item Det er tilstrækkeligt at vise udsagnene for $\rho_\infty$, da enhver $\rho_2$-kugle indeholder en $\rho_\infty$-kugle. For $\rationals^d$, hvis $\underline{x} = (x_1, \ldots, x_d) \in \rationals^d$, vælg da $q_1, \ldots, q_d \in \rationals$ med $q_j \in (x_j - r, x_j + r)$. Hvis $\underline{q} = (q_1, \ldots, q_d)$, så er $\underline{q} \in b_\infty(\underline{x},r)$.

    Tilfældet $(\reals \setminus \rationals)^d$ vises tilsvarende, og det sidste tilfælde følger eftersom $(\reals \setminus \rationals)^d \subseteq \reals^d \setminus \rationals^d$.
\end{solutionsec}
\end{opgavebreak}


\begin{opgave}[1.17]
    Vi nøjes med at vise del (a), idet de andre to delopgaver vises tilsvarende.

    Antag først at $\lim_{t \to a} F(t) = c$, og lad $(t_n)$ være en følge i $I \setminus \{a\}$ som opfylder at $\lim_{n \to \infty} t_n = a$. Lad $\epsilon > 0$, og vælg $\delta > 0$ så $\abs{F(t) - c} < \epsilon$ for $t \in I \setminus \{a\}$ med $\abs{t - a} < \delta$. Vælg nu $N \in \naturals$ så $n \geq N$ medfører at $\abs{t_n - a} < \delta$. For $n \geq N$ følger det da at $\abs{F(t_n) - c} < \epsilon$, hvilket viser at $\lim_{n \to \infty} F(t_n) = c$ som ønsket.

    Den omvendte implikation vises ved kontraponering. Antag derfor at $F(t)$ \emph{ikke} konvergerer mod $c$ for $t \to a$. Der findes da et $\epsilon > 0$ så der for ethvert $\delta > 0$ findes et $t \in I \setminus \{a\}$ således at $\abs{t - a} < \delta$ og $\abs{F(t) - c} \geq \epsilon$. Lad sådan et $\epsilon$ være givet. For ethvert $n \in \naturals$ findes da et $t_n \in I \setminus \{a\}$ så $\abs{t_n - a} < \frac{1}{n}$ og $\abs{F(t_n) - c} \geq \epsilon$. Det følger da at $t_n \to a$ men $F(t_n) \not\to c$.
\end{opgave}

\end{document}