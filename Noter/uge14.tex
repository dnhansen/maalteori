\newcommand{\doctitle}{Uge 14}
\newcommand{\docauthor}{Danny Nygård Hansen}

\documentclass[a4paper, 11pt, article, danish, oneside]{memoir}
\usepackage[utf8]{inputenc}
\usepackage[T1]{fontenc}
\usepackage[danish]{babel}
\usepackage[autostyle, danish=guillemets]{csquotes}

\usepackage[final]{microtype}
\frenchspacing
\raggedbottom

\usepackage{mathtools}
\usepackage{amssymb}
\usepackage[largesmallcaps]{kpfonts}
\linespread{1.06}
\DeclareMathAlphabet\mathfrak{U}{euf}{m}{n}
\SetMathAlphabet\mathfrak{bold}{U}{euf}{b}{n}
\usepackage{inconsolata}

\usepackage{hyperref}
\hypersetup{%
	pdftitle=\doctitle,
	pdfauthor={\docauthor},
    hidelinks,
}

\usepackage{enumitem}
\setenumerate[0]{label=\normalfont(\arabic*)}
\setlist{
	listparindent=\parindent,
	parsep=0pt,
}
\usepackage{array}

\title{\doctitle}
\author{\docauthor}

\newcommand{\overbar}[3]{\mkern #1mu\overline{\mkern-#1mu#3\mkern-#2mu}\mkern #2mu}
\newcommand{\naturals}{\mathbb{N}}
\newcommand{\ints}{\mathbb{Z}}
\newcommand{\rationals}{\mathbb{Q}}
\newcommand{\reals}{\mathbb{R}}
\newcommand{\extreals}{\overbar{1.5}{1.5}{\reals}}
\newcommand{\complex}{\mathbb{C}}


\usepackage{pgffor}

\newcommand{\rvar}[1]{\mathsf{#1}}

\foreach \x in {A,...,Z}{%
    \expandafter\xdef\csname cal\x\endcsname{\noexpand\mathcal{\x}}
    \expandafter\xdef\csname frak\x\endcsname{\noexpand\mathfrak{\x}}
    \expandafter\xdef\csname rand\x\endcsname{\noexpand\rvar{\x}}
}


\usepackage{etoolbox}
\newcommand{\blank}{\mathrel{\;\cdot\;}}
\newcommand{\blankifempty}[1]{\ifstrempty{#1}{\blank}{#1}}
\DeclarePairedDelimiter{\auxdelimlvert}{\lvert}{\rvert}
\DeclarePairedDelimiter{\auxdelimlVert}{\lVert}{\rVert}
\DeclarePairedDelimiterX{\auxdelimanglescomma}[2]{\langle}{\rangle}{#1,#2}
\newcommand{\abs}[2][]{\auxdelimlvert[#1]{\blankifempty{#2}}}
\newcommand{\norm}[2][]{\auxdelimlVert[#1]{\blankifempty{#2}}}
\newcommand{\inner}[2]{\auxdelimanglescomma{\blankifempty{#1}}{\blankifempty{#2}}}


\DeclarePairedDelimiter{\auxdelimparen}{(}{)}
\DeclarePairedDelimiterX{\auxdelimparencomma}[2]{(}{)}{#1,#2}
\DeclarePairedDelimiter{\auxdelimbracket}{[}{]}
\DeclarePairedDelimiterX{\auxdelimbracketcomma}[2]{[}{]}{#1,#2}
\newcommand{\powerset}[2][]{\calP\auxdelimparen[#1]{#2}}
\newcommand{\borel}[2][]{\calB\auxdelimparen[#1]{#2}}
\newcommand{\meas}[2][]{\calM\auxdelimparen[#1]{#2}}
\newcommand{\measC}[2][]{\calM_\complex\auxdelimparen[#1]{#2}}
\newcommand{\measpos}[2][]{\meas[#1]{#2}^+}
\newcommand{\measbound}[2][]{\calM_b\auxdelimparen[#1]{#2}}
\newcommand{\measboundpos}[2][]{\measbound[#1]{#2}^+}


\newcommand{\extmeas}[2][]{\overbar{4.5}{0.5}{\calM}\auxdelimparen[#1]{#2}}
\newcommand{\extmeaspos}[2][]{\extmeas[#1]{#2}^+}
\newcommand{\simplemeas}[2][]{\calS\!\calM\auxdelimparen[#1]{#2}}
\newcommand{\simplemeaspos}[2][]{\simplemeas[#1]{#2}^+}
\newcommand{\sigmaalg}[2][]{\sigma\auxdelimparen[#1]{#2}}
\newcommand{\deltasys}[2][]{\delta\auxdelimparen[#1]{#2}}

\newcommand{\expval}[2][]{\mathbb{E}\auxdelimbracket[#1]{#2}}
\newcommand{\var}[2][]{\operatorname{Var}\auxdelimbracket[#1]{#2}}
\newcommand{\cov}[3][]{\operatorname{Cov}\auxdelimbracketcomma[#1]{#2}{#3}}


\renewcommand{\Re}{\operatorname{Re}}
\renewcommand{\Im}{\operatorname{Im}}
\newcommand{\conj}[1]{\overline{#1}}
\newcommand{\dif}{\mathop{}\!\mathrm{d}}
\DeclareMathOperator{\id}{id}
\newcommand{\indicator}[1]{\mathbf{1}_{#1}}

% Lattice operations
\newcommand{\meet}{\land}
\newcommand{\join}{\lor}

\DeclareMathOperator*{\smallbigvee}{\textstyle\bigvee}
\DeclareMathOperator*{\bigjoin}{\mathchoice
    {\smallbigvee}%
    {\bigvee}%
    {\bigvee}%
    {\bigvee}%
}
\DeclareMathOperator*{\smallbigwedge}{\textstyle\bigwedge}
\DeclareMathOperator*{\bigmeet}{\mathchoice
    {\smallbigwedge}%
    {\bigwedge}%
    {\bigwedge}%
    {\bigwedge}%
}



\newcommand*\union\cup
\newcommand*\intersect\cap

\DeclareMathOperator*{\smallbigcup}{\textstyle\bigcup}
\DeclareMathOperator*{\bigunion}{\mathchoice
    {\smallbigcup}%
    {\bigcup}%
    {\bigcup}%
    {\bigcup}%
}
\DeclareMathOperator*{\smallbigcap}{\textstyle\bigcap}
\DeclareMathOperator*{\bigintersect}{\mathchoice
    {\smallbigcap}%
    {\bigcap}%
    {\bigcap}%
    {\bigcap}%
}


\DeclarePairedDelimiterX{\set}[2]{\lbrace}{\rbrace}{#1\;\delimsize\vert\;#2}

\newcommand{\defeq}{\coloneqq}
\newcommand{\eqdef}{\eqqcolon}
\renewcommand{\phi}{\varphi}
\newcommand{\iu}{\mathrm{i}\mkern1mu}
\DeclareMathOperator{\e}{\mathrm{e}}

\newcommand{\ball}[3][]{%
    \ifstrempty{#1}%
        {%
            b\auxdelimparencomma{#2}{#3}%
        }{%
            b_{#1}\auxdelimparencomma{#2}{#3}%
        }%
}

\newcommand{\converges}[1]{\xrightarrow[#1]{}}
\DeclareMathOperator{\supp}{supp}
\let\oldvec\vec
\renewcommand{\vec}[1]{\underline{#1}}
\newcommand{\Tr}[1][]{%
    \ifstrempty{#1}%
        {%
            \operatorname{Tr}%
        }{%
            \operatorname{Tr}_{#1}%
        }%
}


\usepackage{listofitems}
\setsepchar{,}

\makeatletter
\newcommand{\mat@dims}[1]{%
    \readlist*\@dims{#1}%
    \ifnum \@dimslen=1
        \def\@dimsout{\@dims[1]}%
    \else
        \def\@dimsout{\@dims[1], \@dims[2]}%
    \fi
    \@dimsout
}


\newcommand{\matgroup}[3]{\mathrm{#1}_{#2}(#3)}
\newcommand{\matGL}[2]{\matgroup{GL}{#1}{#2}}
\newcommand{\trans}{^{\top}}
\newcommand{\mat}[2]{M_{\mat@dims{#1}}(#2)}

\makeatother

\DeclareMathOperator{\Span}{span}
\DeclareMathOperator{\clSpan}{\overbar{0.5}{1.5}{span}}

\newcommand\inv{^{-1}}
\newcommand{\preim}[2][]{^{-1}\auxdelimparen[#1]{#2}}

\newcommand{\dsupp}[2][]{\mathrm{Sp}_d\auxdelimparen[#1]{#2}}

\usepackage[amsmath,thmmarks,hyperref]{ntheorem}
\usepackage{bbding}

\newcommand{\theorembullet}{{\footnotesize\textbullet}}
\newcommand{\pencilsymbol}{\raisebox{-2pt}{\normalfont\PencilLeft}}
\makeatletter
\newtheoremstyle{changedotcustomnumber}%
    {}%
    {\item[\hskip\labelsep \theorem@headerfont ##3~~\theorembullet~~##1\theorem@separator]}
\newtheoremstyle{changedotbreakcustomnumber}%
    {}%
    {\item[\rlap{\vbox{\hbox{\hskip\labelsep \theorem@headerfont
            ##3~~\theorembullet~~##1\theorem@separator}\hbox{\strut}}}]}
\makeatother

\theorembodyfont{\normalfont}
\theoremseparator{~~}
\theoremsymbol{\ensuremath{\blacksquare}}
\theoremstyle{changedotcustomnumber}
\newtheorem{opgave}{\pencilsymbol}
\theoremstyle{changedotbreakcustomnumber}
\newtheorem{opgavebreak}{\pencilsymbol}

\newlist{solutionsec}{enumerate}{1}
\setlist[solutionsec]{leftmargin=0pt, parsep=0pt, listparindent=\parindent, label=(\alph*), labelsep=0pt, labelwidth=20pt, itemindent=20pt, align=left, itemsep=.5\baselineskip}


\begin{document}

\maketitle

% • 12.5, 13.2, 12.7.
% • 13.4, 13.5.

\begin{opgavebreak}[12.5]
\begin{solutionsec}
    \item Bemærk at $f',f'' \in C_c(\reals,\complex) \subseteq \calL^1_\complex(\lambda)$, og at
    %
    \begin{equation*}
        \widehat{f''}(t)
            = \iu t \widehat{f'}(t)
            = - t^2 \hat{f}(t)
    \end{equation*}
    %
    ved Sætning~12.1.6(ii). Men $\widehat{f''}$ er begrænset ved Sætning~12.1.3(i).

    \item Da $\hat{f}$ er begrænset, er funktionen $t \mapsto (t^2+1) \hat{f}(t)$ også begrænset, sig af $R > 0$. Men så er
    %
    \begin{equation*}
        \abs{\hat{f}(t)}
            \leq \frac{R}{t^2+1},
    \end{equation*}
    %
    og funktionen (af $t$) på højre side ligger i $\calL^1_\complex(\lambda)$ (sæt f.eks. $\sigma = 1$ i Opgave~12.4(b)). Det ønskede følger da af Opgave~12.3 da $f$ er kontinuert.
\end{solutionsec}
\end{opgavebreak}


\begin{opgave}[13.2]
    Sæt $c = \inf \{F_\randX = 1\}$ og bemærk at $c \in \reals$, da der både må findes $t \in \reals$ så $P(\randX \leq t) = 0$ og $P(\randX \leq t) = 1$. Da $F_\randX$ er højrekontinuert, er $F_\randX(c) = 1$, men omvendt er $F_\randX(t) = 0$ for $t < c$. Altså er $F_\randX$ også fordelingsfunktionen for Diracmålet $\delta_c$, så $P_\randX = \delta_c$ ved 13.1.5.
\end{opgave}


\begin{opgavebreak}[12.7]
\begin{solutionsec}
    \item Oplagt.

    \item Bemærk at
    %
    \begin{equation*}
        H_1 * u(2x)
            = \int_\reals \e^{-\abs{2x-y}} \e^{-\abs{y}} \,\lambda(\dif y)
            = \int_\reals \e^{-(\abs{x-y}+\abs{x+y})} \,\lambda(\dif y).
    \end{equation*}
    %
    Altså er $H_1 * u$ lige, så vi kan antage $x \geq 0$. Integranden er også lige i $y$ for ethvert $x$, så for $x \geq 0$ er
    %
    \begin{align*}
        H_1 * u(2x)
            &= 2\int_0^\infty \e^{-(\abs{x-y}+x+y)} \,\lambda(\dif y)
             = 2\int_0^\infty \e^{-2(x \join y)} \,\lambda(\dif y) \\
            &= 2\int_0^x \e^{-2x} \,\lambda(\dif y) + 2\int_x^\infty \e^{-2y} \,\lambda(\dif y) \\
            &= (2x+1) \e^{-2x}.
    \end{align*}
    %
    Dermed er $H_1 * u(x) = (x+1)\e^{-x}$ for $x \geq 0$.

    \item Erstat $u$ med $u \indicator{(-R,R)}$ for $R > 0$. (Bemærk at differentiation er defineret lokalt. Sammenlign Opgave~5.24(e).)
\end{solutionsec}
\end{opgavebreak}


\begin{opgavebreak}[13.4]
\begin{solutionsec}
    \item Betragt $(\reals,\borel{\reals},\mu_1)$ og lad $\randX_1$ være identitetsfunktionen på $\reals$.

    \item Sæt $\mu = \mu_1 \otimes \cdots \otimes \mu_n$ og betragt produktrummet $(\reals^n, \borel{\reals^n}, \mu)$, og lad $\randX_j$ være den $j$'te koordinatprojektion $(x_1, \ldots, x_n) \mapsto x_j$. For uafhængighed, bemærk at $(\randX_1, \ldots, \randX_n)$ blot er identitetsfunktionen på $\reals^n$, så dens fordeling er $\mu$. Benyt da Sætning~13.5.3.

    Vi bemærker at dette specielt viser at hvis $\randX$ og $\randY$ er stokastiske variable, da findes et sandsynlighedsfelt $(\Omega,\calF,P)$ hvorpå der findes uafhængige \enquote{kopier} af $\randX$ og $\randY$, dvs. stokatiske variable $\tilde{\randX}$ og $\tilde{\randY}$ med $P_\randX = P_{\tilde{\randX}}$ og $P_\randY = P_{\tilde{\randY}}$. Hvis $\randX$ og $\randY$ er defineret på samme sandsynlighedsfelt, da kan man altså altid finde \emph{uafhængige} kopier deraf. Dette er især nyttigt idet vi i sandsynlighedsteori ofte kun er interesseret i \emph{fordelingen} af stokatiske variable, jf. eksempelvis Opgave~13.5.
\end{solutionsec}
\end{opgavebreak}


\begin{opgavebreak}[13.5]
\begin{solutionsec}
    \item Oplagt.

    \item Benyt vinket og lad f.eks. $\randX = \indicator{[0,1/2]}$ og $\randY = \indicator{[1/2,1]}$.

    \item Bemærk at
    %
    \begin{equation*}
        \expval{\abs{\randX}^p}
            = \int_\Omega \abs{\randX}^p \dif P
            = \int_\reals \abs{x}^p P_\randX(x)
            = \int_\reals \abs{y}^p P_\randY(y)
            = \int_\Omega \abs{\randY}^p \dif P
            = \expval{\abs{\randY}^p}.
    \end{equation*}

    \item Udeladt.
\end{solutionsec}
\end{opgavebreak}


\end{document}
