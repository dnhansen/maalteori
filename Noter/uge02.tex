\newcommand{\doctitle}{Uge 2}
\newcommand{\docauthor}{Danny Nygård Hansen}

\documentclass[a4paper, 11pt, article, danish, oneside]{memoir}
\usepackage[utf8]{inputenc}
\usepackage[T1]{fontenc}
\usepackage[danish]{babel}
\usepackage[autostyle, danish=guillemets]{csquotes}

\usepackage[final]{microtype}
\frenchspacing
\raggedbottom

\usepackage{mathtools}
\usepackage{amssymb}
\usepackage[largesmallcaps]{kpfonts}
\linespread{1.06}
\DeclareMathAlphabet\mathfrak{U}{euf}{m}{n}
\SetMathAlphabet\mathfrak{bold}{U}{euf}{b}{n}
\usepackage{inconsolata}

\usepackage{hyperref}
\hypersetup{%
	pdftitle=\doctitle,
	pdfauthor={\docauthor},
    hidelinks,
}

\usepackage{enumitem}
\setenumerate[0]{label=\normalfont(\arabic*)}
\setlist{
	listparindent=\parindent,
	parsep=0pt,
}
\usepackage{array}

\title{\doctitle}
\author{\docauthor}

\newcommand{\overbar}[3]{\mkern #1mu\overline{\mkern-#1mu#3\mkern-#2mu}\mkern #2mu}
\newcommand{\naturals}{\mathbb{N}}
\newcommand{\ints}{\mathbb{Z}}
\newcommand{\rationals}{\mathbb{Q}}
\newcommand{\reals}{\mathbb{R}}
\newcommand{\extreals}{\overbar{1.5}{1.5}{\reals}}
\newcommand{\complex}{\mathbb{C}}


\usepackage{pgffor}

\newcommand{\rvar}[1]{\mathsf{#1}}

\foreach \x in {A,...,Z}{%
    \expandafter\xdef\csname cal\x\endcsname{\noexpand\mathcal{\x}}
    \expandafter\xdef\csname frak\x\endcsname{\noexpand\mathfrak{\x}}
    \expandafter\xdef\csname rand\x\endcsname{\noexpand\rvar{\x}}
}


\usepackage{etoolbox}
\newcommand{\blank}{\mathrel{\;\cdot\;}}
\newcommand{\blankifempty}[1]{\ifstrempty{#1}{\blank}{#1}}
\DeclarePairedDelimiter{\auxdelimlvert}{\lvert}{\rvert}
\DeclarePairedDelimiter{\auxdelimlVert}{\lVert}{\rVert}
\DeclarePairedDelimiterX{\auxdelimanglescomma}[2]{\langle}{\rangle}{#1,#2}
\newcommand{\abs}[1]{\auxdelimlvert{\blankifempty{#1}}}
\newcommand{\norm}[1]{\auxdelimlVert{\blankifempty{#1}}}
\newcommand{\inner}[2]{\auxdelimanglescomma{\blankifempty{#1}}{\blankifempty{#2}}}


\DeclarePairedDelimiter{\auxdelimparen}{(}{)}
\DeclarePairedDelimiterX{\auxdelimparencomma}[2]{(}{)}{#1,#2}
\DeclarePairedDelimiter{\auxdelimbracket}{[}{]}
\DeclarePairedDelimiterX{\auxdelimbracketcomma}[2]{[}{]}{#1,#2}
\newcommand{\powerset}[2][]{\calP\auxdelimparen[#1]{#2}}
\newcommand{\borel}[2][]{\calB\auxdelimparen[#1]{#2}}
\newcommand{\meas}[2][]{\calM\auxdelimparen[#1]{#2}}
\newcommand{\measC}[2][]{\calM_\complex\auxdelimparen[#1]{#2}}
\newcommand{\measpos}[2][]{\meas[#1]{#2}^+}
\newcommand{\measbound}[2][]{\calM_b\auxdelimparen[#1]{#2}}
\newcommand{\measboundpos}[2][]{\measbound[#1]{#2}^+}


\newcommand{\extmeas}[2][]{\overbar{4.5}{0.5}{\calM}\auxdelimparen[#1]{#2}}
\newcommand{\extmeaspos}[2][]{\extmeas[#1]{#2}^+}
\newcommand{\simplemeas}[2][]{\calS\!\calM\auxdelimparen[#1]{#2}}
\newcommand{\simplemeaspos}[2][]{\simplemeas[#1]{#2}^+}
\newcommand{\sigmaalg}[2][]{\sigma\auxdelimparen[#1]{#2}}
\newcommand{\deltasys}[2][]{\delta\auxdelimparen[#1]{#2}}

\newcommand{\expval}[2][]{\mathbb{E}\auxdelimbracket[#1]{#2}}
\newcommand{\var}[2][]{\operatorname{Var}\auxdelimbracket[#1]{#2}}
\newcommand{\cov}[3][]{\operatorname{Cov}\auxdelimbracketcomma[#1]{#2}{#3}}


\renewcommand{\Re}{\operatorname{Re}}
\renewcommand{\Im}{\operatorname{Im}}
\newcommand{\conj}[1]{\overline{#1}}
\newcommand{\dif}{\mathop{}\!\mathrm{d}}
\DeclareMathOperator{\id}{id}
\newcommand{\indicator}[1]{\mathbf{1}_{#1}}

% Lattice operations
\newcommand{\meet}{\land}
\newcommand{\join}{\lor}

\DeclareMathOperator*{\smallbigvee}{\textstyle\bigvee}
\DeclareMathOperator*{\bigjoin}{\mathchoice
    {\smallbigvee}%
    {\bigvee}%
    {\bigvee}%
    {\bigvee}%
}
\DeclareMathOperator*{\smallbigwedge}{\textstyle\bigwedge}
\DeclareMathOperator*{\bigmeet}{\mathchoice
    {\smallbigwedge}%
    {\bigwedge}%
    {\bigwedge}%
    {\bigwedge}%
}



\newcommand*\union\cup
\newcommand*\intersect\cap

\DeclareMathOperator*{\smallbigcup}{\textstyle\bigcup}
\DeclareMathOperator*{\bigunion}{\mathchoice
    {\smallbigcup}%
    {\bigcup}%
    {\bigcup}%
    {\bigcup}%
}
\DeclareMathOperator*{\smallbigcap}{\textstyle\bigcap}
\DeclareMathOperator*{\bigintersect}{\mathchoice
    {\smallbigcap}%
    {\bigcap}%
    {\bigcap}%
    {\bigcap}%
}


\DeclarePairedDelimiterX{\set}[2]{\lbrace}{\rbrace}{#1\;\delimsize\vert\;#2}

\newcommand{\defeq}{\coloneqq}
\renewcommand{\phi}{\varphi}
\newcommand{\iu}{\mathrm{i}\mkern1mu}
\DeclareMathOperator{\e}{\mathrm{e}}

\newcommand{\ball}[3][]{%
    \ifstrempty{#1}%
        {%
            b\auxdelimparencomma{#2}{#3}%
        }{%
            b_{#1}\auxdelimparencomma{#2}{#3}%
        }%
}

\newcommand{\converges}[1]{\xrightarrow[#1]{}}
\DeclareMathOperator{\supp}{supp}
\let\oldvec\vec
\renewcommand{\vec}[1]{\underline{#1}}
\newcommand{\Tr}[1][]{%
    \ifstrempty{#1}%
        {%
            \operatorname{Tr}%
        }{%
            \operatorname{Tr}_{#1}%
        }%
}


\usepackage{listofitems}
\setsepchar{,}

\makeatletter
\newcommand{\mat@dims}[1]{%
    \readlist*\@dims{#1}%
    \ifnum \@dimslen=1
        \def\@dimsout{\@dims[1]}%
    \else
        \def\@dimsout{\@dims[1], \@dims[2]}%
    \fi
    \@dimsout
}


\newcommand{\matgroup}[3]{\mathrm{#1}_{#2}(#3)}
\newcommand{\matGL}[2]{\matgroup{GL}{#1}{#2}}
\newcommand{\trans}{^{\top}}
\newcommand{\mat}[2]{M_{\mat@dims{#1}}(#2)}

\makeatother

\DeclareMathOperator{\Span}{span}
\DeclareMathOperator{\clSpan}{\overbar{0.5}{1.5}{span}}

\newcommand\inv{^{\langle-1\rangle}}
\newcommand{\preim}[2][]{^{-1}\auxdelimparen[#1]{#2}}

\newcommand{\dsupp}[2][]{\mathrm{Sp}_d\auxdelimparen[#1]{#2}}

\usepackage[amsmath,thmmarks,hyperref]{ntheorem}
\usepackage{bbding}

\newcommand{\theorembullet}{{\footnotesize\textbullet}}
\newcommand{\pencilsymbol}{\raisebox{-2pt}{\normalfont\PencilLeft}}
\makeatletter
\newtheoremstyle{changedotcustomnumber}%
    {}%
    {\item[\hskip\labelsep \theorem@headerfont ##3~~\theorembullet~~##1\theorem@separator]}
\newtheoremstyle{changedotbreakcustomnumber}%
    {}%
    {\item[\rlap{\vbox{\hbox{\hskip\labelsep \theorem@headerfont
            ##3~~\theorembullet~~##1\theorem@separator}\hbox{\strut}}}]}
\makeatother

\theorembodyfont{\normalfont}
\theoremseparator{~~}
\theoremsymbol{\ensuremath{\blacksquare}}
\theoremstyle{changedotcustomnumber}
\newtheorem{opgave}{\pencilsymbol}
\theoremstyle{changedotbreakcustomnumber}
\newtheorem{opgavebreak}{\pencilsymbol}

\newlist{solutionsec}{enumerate}{1}
\setlist[solutionsec]{leftmargin=0pt, parsep=0pt, listparindent=\parindent, label=(\alph*), labelsep=0pt, labelwidth=20pt, itemindent=20pt, align=left, itemsep=.5\baselineskip}


\begin{document}

\maketitle

\begin{opgavebreak}[1.3]
\begin{solutionsec}
    \item Alle åbne intervaller er selvfølgelig Borelmængder. Ethvert lukket interval er på en af formerne
    %
    \begin{align*}
        [a,b]
            &= \bigl( (-\infty,a) \union (b,\infty) \bigr)^c, \\
        (-\infty,b]
            &= (b,\infty)^c, \\
        [a,\infty)
            &= (-\infty,a)^c,
    \end{align*}
    %
    så disse er også Borelmængder.\footnote{Vi kan endda nøjes med endeligt mange mængdeoperationer! Bemærk at $[a,a] = \{a\}$ også er et interval.} Dernæst er eksempelvis
    %
    \begin{equation*}
        (a,b]
            = (a,b) \union \{b\}
            = (a,b) \union [b,b].
    \end{equation*}

    \item Enhver etpunktsmængde er en Borelmængde pr. del (a), og en tællelig mængde er jo en tællelig forening af etpunktsmængder.
\end{solutionsec}
\end{opgavebreak}

\begin{opgave}[1.4]
    Bemærk først at alle mængderne i nævnte mængdesystemer er Borelmængder. Hvis omvendt $\calQ$ betegner samlingen af åbne intervaller $(a,b)$ hvor $a,b \in \rationals$, er det pr. Sætning~1.2.4 tilstrækkeligt at vise at $\calQ \subseteq \sigma(\calF)$, $\calQ \subseteq \sigma(\calK)$, osv. Vi kan eksempelvis skrive
    %
    \begin{equation*}
        (a,b)
            = \bigintersect_{n\in\naturals} [a+\tfrac{1}{n},b-\tfrac{1}{n}],
    \end{equation*}
    %
    og alle de lukkede og begrænsede intervaller er kompakte, så $(a,b) \in \sigma(\calK)$.
\end{opgave}

\begin{opgave}[1.5]
    Bemærk at en tælleligt uendelig forening $A$ af endelige mængder ikke (nødvendigvis) er endelig, men den er samtidigt selv tællelig, så hverken $A$ eller $A^c$ ligger i $\calA$. Altså er $\calA$ ikke en $\sigma$-algebra.

    For at vise at $\calA$ er en algebra, bemærk at hvis $A,B \in \calA$ med $A$ endelig og $B^c$ endelig, da er
    %
    \begin{equation*}
        (A \union B)^c
            = A^c \intersect B^c
            \subseteq B^c.
    \end{equation*}
\end{opgave}

\begin{opgavebreak}[1.7]
\begin{solutionsec}
    \item Vis at
    %
    \begin{equation*}
        \{A_1,A_2,A_3\} \subseteq \sigma(\{\{1\},\{2\},\{3,4\}\})
    \end{equation*}
    %
    og omvendt at
    %
    \begin{equation*}
        \{\{1\},\{2\},\{3,4\}\} \subseteq \sigma(\{A_1,A_2,A_3\}).
    \end{equation*}
    %
    Man kan eksplicit bestemme mængderne i $\sigma$-algebraen på (mindst) to forskellige måder:

    For det første, bemærk at $X$ er endelig, så $\powerset{X}$ er også endelig, og $\sigma$-algebraen er derfor også endelig. Altså er det tilstrækkeligt at se på endelige mængdeoperationer. Tilføj først $X$ til $\sigma$-algebraen, og udfør derefter de relevante mængdeoperationer (foreningsmængde og komplementering) successivt på alle hidtil opnåede mængder. Når det ikke er muligt at danne nye mængder, må de opnåede mængder udgøre en algebra, og derfor en $\sigma$-algebra i $X$, og denne er klart den mindste der indeholder mængderne vi startede med.

    For det andet kan man overveje at $\{\{1\},\{2\},\{3,4\}\}$ udgør en partition af $X$, så den heraf frembragte $\sigma$-algebra er en partitions-$\sigma$-algebra. Mængderne i $\sigma$-algebraen er derfor alle mulige foreninger af mængder fra $\{\{1\},\{2\},\{3,4\}\}$ (jf. Lemma~A.6.2), hvilket inkluderer den tomme mængde som blot er foreningen af \emph{ingen} mængder.
    
    \item Selvom $\naturals$ ikke er endelig, er det alligevel klart at $\sigma$-algebraen er endelig, så den første metode fra del (a) kan stadig benyttes.\footnote{Vi behøver ikke på forhånd vide at $\sigma$-algebraen er endelig, det opdager vi når vi løber tør for nye mængder at frembringe.} Bemærk alternativt at
    %
    \begin{equation*}
        \sigma(\{A_1,A_2,A_3\})
            = \sigma(\{\{1\},\{2\},\{3,4\},\{5,6,\ldots\}\}),
    \end{equation*}
    %
    og at $\{\{1\},\{2\},\{3,4\},\{5,6,\ldots\}\}$ er en partition af $\naturals$, og gå derefter frem som i den anden metode fra (a).
\end{solutionsec}
\end{opgavebreak}

\begin{opgavebreak}[1.9]
\begin{solutionsec}
    \item Bemærk at $x \in \liminf_{n\to\infty} A_n$ hvis og kun hvis $x \in \bigintersect_{k=n}^\infty A_k$ for et $n \in \naturals$. Dette siger præcis at $x$ ligger i alle $A_n$ for $n$ stor nok.

    Tilsvarende har vi $x \in \limsup_{n\to\infty} A_n$ hvis og kun hvis $x \in \bigunion_{k=n}^\infty A_k$ for alle $n \in \naturals$. Hvis der var et største $N$ så $x \in A_N$, da ville $x$ ikke ligge i $\bigunion_{k=N+1}^\infty A_k$, hvilket er en modstrid.

    \item F.eks. er $\bigintersect_{n=1}^\infty A_n \subseteq \bigintersect_{k=n}^\infty A_k$. Det følger da af definitionerne på $\liminf$ og $\limsup$.

    \item Hvis $(A_n)$ er voksende, er $\bigintersect_{k=n}^\infty A_k = A_n$ og $\bigunion_{k=n}^\infty A_k = \bigunion_{k=1}^\infty A_k$. Det følger da af definitionerne.

    \item Dette kan vises direkte, men det følger også af (c) ved at tage komplementer.

    \item Benyt karakterisationen fra (a): Hvis $x \in [0,\limsup_{n\to\infty} x_n)$, da er
    %
    \begin{equation*}
        x
            < \limsup_{n\to\infty} x_n
            \leq \sup_{k \geq n} x_k
    \end{equation*}
    %
    for alle $n \in \naturals$. Dvs. at $x \leq x_n$ for uendeligt mange $n$. Hvis $x \in \limsup_{n\to\infty} A_n$, så er $x \leq x_n$ for uendeligt mange $n$, og da er $x \leq \sup_{k \geq n} x_k$ for alle $n$.
\end{solutionsec}
\end{opgavebreak}

\begin{opgave}[1.11]
    For at vise tællelig additivitet af $\delta_a$, husk at mængderne i foreningen er disjunkte, så højst én af dem kan indeholde $a$.
\end{opgave}

\end{document}