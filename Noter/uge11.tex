\newcommand{\doctitle}{Uge 11}
\newcommand{\docauthor}{Danny Nygård Hansen}

\documentclass[a4paper, 11pt, article, danish, oneside]{memoir}
\usepackage[utf8]{inputenc}
\usepackage[T1]{fontenc}
\usepackage[danish]{babel}
\usepackage[autostyle, danish=guillemets]{csquotes}

\usepackage[final]{microtype}
\frenchspacing
\raggedbottom

\usepackage{mathtools}
\usepackage{amssymb}
\usepackage[largesmallcaps]{kpfonts}
\linespread{1.06}
\DeclareMathAlphabet\mathfrak{U}{euf}{m}{n}
\SetMathAlphabet\mathfrak{bold}{U}{euf}{b}{n}
\usepackage{inconsolata}

\usepackage{hyperref}
\hypersetup{%
	pdftitle=\doctitle,
	pdfauthor={\docauthor},
    hidelinks,
}

\usepackage{enumitem}
\setenumerate[0]{label=\normalfont(\arabic*)}
\setlist{
	listparindent=\parindent,
	parsep=0pt,
}
\usepackage{array}

\title{\doctitle}
\author{\docauthor}

\newcommand{\overbar}[3]{\mkern #1mu\overline{\mkern-#1mu#3\mkern-#2mu}\mkern #2mu}
\newcommand{\naturals}{\mathbb{N}}
\newcommand{\ints}{\mathbb{Z}}
\newcommand{\rationals}{\mathbb{Q}}
\newcommand{\reals}{\mathbb{R}}
\newcommand{\extreals}{\overbar{1.5}{1.5}{\reals}}
\newcommand{\complex}{\mathbb{C}}


\usepackage{pgffor}

\newcommand{\rvar}[1]{\mathsf{#1}}

\foreach \x in {A,...,Z}{%
    \expandafter\xdef\csname cal\x\endcsname{\noexpand\mathcal{\x}}
    \expandafter\xdef\csname frak\x\endcsname{\noexpand\mathfrak{\x}}
    \expandafter\xdef\csname rand\x\endcsname{\noexpand\rvar{\x}}
}


\usepackage{etoolbox}
\newcommand{\blank}{\mathrel{\;\cdot\;}}
\newcommand{\blankifempty}[1]{\ifstrempty{#1}{\blank}{#1}}
\DeclarePairedDelimiter{\auxdelimlvert}{\lvert}{\rvert}
\DeclarePairedDelimiter{\auxdelimlVert}{\lVert}{\rVert}
\DeclarePairedDelimiterX{\auxdelimanglescomma}[2]{\langle}{\rangle}{#1,#2}
\newcommand{\abs}[2][]{\auxdelimlvert[#1]{\blankifempty{#2}}}
\newcommand{\norm}[1]{\auxdelimlVert{\blankifempty{#1}}}
\newcommand{\inner}[2]{\auxdelimanglescomma{\blankifempty{#1}}{\blankifempty{#2}}}


\DeclarePairedDelimiter{\auxdelimparen}{(}{)}
\DeclarePairedDelimiterX{\auxdelimparencomma}[2]{(}{)}{#1,#2}
\DeclarePairedDelimiter{\auxdelimbracket}{[}{]}
\DeclarePairedDelimiterX{\auxdelimbracketcomma}[2]{[}{]}{#1,#2}
\newcommand{\powerset}[2][]{\calP\auxdelimparen[#1]{#2}}
\newcommand{\borel}[2][]{\calB\auxdelimparen[#1]{#2}}
\newcommand{\meas}[2][]{\calM\auxdelimparen[#1]{#2}}
\newcommand{\measC}[2][]{\calM_\complex\auxdelimparen[#1]{#2}}
\newcommand{\measpos}[2][]{\meas[#1]{#2}^+}
\newcommand{\measbound}[2][]{\calM_b\auxdelimparen[#1]{#2}}
\newcommand{\measboundpos}[2][]{\measbound[#1]{#2}^+}


\newcommand{\extmeas}[2][]{\overbar{4.5}{0.5}{\calM}\auxdelimparen[#1]{#2}}
\newcommand{\extmeaspos}[2][]{\extmeas[#1]{#2}^+}
\newcommand{\simplemeas}[2][]{\calS\!\calM\auxdelimparen[#1]{#2}}
\newcommand{\simplemeaspos}[2][]{\simplemeas[#1]{#2}^+}
\newcommand{\sigmaalg}[2][]{\sigma\auxdelimparen[#1]{#2}}
\newcommand{\deltasys}[2][]{\delta\auxdelimparen[#1]{#2}}

\newcommand{\expval}[2][]{\mathbb{E}\auxdelimbracket[#1]{#2}}
\newcommand{\var}[2][]{\operatorname{Var}\auxdelimbracket[#1]{#2}}
\newcommand{\cov}[3][]{\operatorname{Cov}\auxdelimbracketcomma[#1]{#2}{#3}}


\renewcommand{\Re}{\operatorname{Re}}
\renewcommand{\Im}{\operatorname{Im}}
\newcommand{\conj}[1]{\overline{#1}}
\newcommand{\dif}{\mathop{}\!\mathrm{d}}
\DeclareMathOperator{\id}{id}
\newcommand{\indicator}[1]{\mathbf{1}_{#1}}

% Lattice operations
\newcommand{\meet}{\land}
\newcommand{\join}{\lor}

\DeclareMathOperator*{\smallbigvee}{\textstyle\bigvee}
\DeclareMathOperator*{\bigjoin}{\mathchoice
    {\smallbigvee}%
    {\bigvee}%
    {\bigvee}%
    {\bigvee}%
}
\DeclareMathOperator*{\smallbigwedge}{\textstyle\bigwedge}
\DeclareMathOperator*{\bigmeet}{\mathchoice
    {\smallbigwedge}%
    {\bigwedge}%
    {\bigwedge}%
    {\bigwedge}%
}



\newcommand*\union\cup
\newcommand*\intersect\cap

\DeclareMathOperator*{\smallbigcup}{\textstyle\bigcup}
\DeclareMathOperator*{\bigunion}{\mathchoice
    {\smallbigcup}%
    {\bigcup}%
    {\bigcup}%
    {\bigcup}%
}
\DeclareMathOperator*{\smallbigcap}{\textstyle\bigcap}
\DeclareMathOperator*{\bigintersect}{\mathchoice
    {\smallbigcap}%
    {\bigcap}%
    {\bigcap}%
    {\bigcap}%
}


\DeclarePairedDelimiterX{\set}[2]{\lbrace}{\rbrace}{#1\;\delimsize\vert\;#2}

\newcommand{\defeq}{\coloneqq}
\newcommand{\eqdef}{\eqqcolon}
\renewcommand{\phi}{\varphi}
\newcommand{\iu}{\mathrm{i}\mkern1mu}
\DeclareMathOperator{\e}{\mathrm{e}}

\newcommand{\ball}[3][]{%
    \ifstrempty{#1}%
        {%
            b\auxdelimparencomma{#2}{#3}%
        }{%
            b_{#1}\auxdelimparencomma{#2}{#3}%
        }%
}

\newcommand{\converges}[1]{\xrightarrow[#1]{}}
\DeclareMathOperator{\supp}{supp}
\let\oldvec\vec
\renewcommand{\vec}[1]{\underline{#1}}
\newcommand{\Tr}[1][]{%
    \ifstrempty{#1}%
        {%
            \operatorname{Tr}%
        }{%
            \operatorname{Tr}_{#1}%
        }%
}


\usepackage{listofitems}
\setsepchar{,}

\makeatletter
\newcommand{\mat@dims}[1]{%
    \readlist*\@dims{#1}%
    \ifnum \@dimslen=1
        \def\@dimsout{\@dims[1]}%
    \else
        \def\@dimsout{\@dims[1], \@dims[2]}%
    \fi
    \@dimsout
}


\newcommand{\matgroup}[3]{\mathrm{#1}_{#2}(#3)}
\newcommand{\matGL}[2]{\matgroup{GL}{#1}{#2}}
\newcommand{\trans}{^{\top}}
\newcommand{\mat}[2]{M_{\mat@dims{#1}}(#2)}

\makeatother

\DeclareMathOperator{\Span}{span}
\DeclareMathOperator{\clSpan}{\overbar{0.5}{1.5}{span}}

\newcommand\inv{^{-1}}
\newcommand{\preim}[2][]{^{-1}\auxdelimparen[#1]{#2}}

\newcommand{\dsupp}[2][]{\mathrm{Sp}_d\auxdelimparen[#1]{#2}}

\usepackage[amsmath,thmmarks,hyperref]{ntheorem}
\usepackage{bbding}

\newcommand{\theorembullet}{{\footnotesize\textbullet}}
\newcommand{\pencilsymbol}{\raisebox{-2pt}{\normalfont\PencilLeft}}
\makeatletter
\newtheoremstyle{changedotcustomnumber}%
    {}%
    {\item[\hskip\labelsep \theorem@headerfont ##3~~\theorembullet~~##1\theorem@separator]}
\newtheoremstyle{changedotbreakcustomnumber}%
    {}%
    {\item[\rlap{\vbox{\hbox{\hskip\labelsep \theorem@headerfont
            ##3~~\theorembullet~~##1\theorem@separator}\hbox{\strut}}}]}
\makeatother

\theorembodyfont{\normalfont}
\theoremseparator{~~}
\theoremsymbol{\ensuremath{\blacksquare}}
\theoremstyle{changedotcustomnumber}
\newtheorem{opgave}{\pencilsymbol}
\theoremstyle{changedotbreakcustomnumber}
\newtheorem{opgavebreak}{\pencilsymbol}

\newlist{solutionsec}{enumerate}{1}
\setlist[solutionsec]{leftmargin=0pt, parsep=0pt, listparindent=\parindent, label=(\alph*), labelsep=0pt, labelwidth=20pt, itemindent=20pt, align=left, itemsep=.5\baselineskip}


\begin{document}

\maketitle

% • 9.2, 9.3, 9.6.
% • 9.5, 9.8.
% • 8.3.


\begin{opgavebreak}[9.2]
\begin{solutionsec}
    \item Uligheden \enquote{$\geq$} følger af Cauchy-Schwarz' ulighed. Den omvendte ulighed følger ved at sætte $v = u/\norm{u}^2$ (hvis $\norm{u} = 0$ gælder der oplagt lighed).

    \item Ja, vi benytter ikke egenskaben (ip4) i del (a).
\end{solutionsec}
\end{opgavebreak}


\begin{opgavebreak}[9.3]
\begin{solutionsec}
    \item Oplagt da et snit af lukkede mængder er lukket, så $\overline{M} \in \calF(M)$.

    \item Følger af at $X \setminus \overline{M}$ er åben, så ethvert punkt deri er centrum i en åben kugle som ikke snitter $M$.

    \item Se på kuglerne $b_\rho(x,\tfrac{1}{n})$ for $n \in \naturals$.

    \item Oplagt.

    \item Hvis f.eks. $u,v \in \overline{U}$, så findes følger $(u_n)$ og $(v_n)$ i $U$ som konvergerer mod $u$ hhv. $v$. Så konvergerer $(u_n + v_n)$ mod $u + v$.
\end{solutionsec}
\end{opgavebreak}


\begin{opgave}[9.6]
    Inklusionen \enquote{$\subseteq$} følger da hvis $u \in U$, så er $u$ ortogonal på alle vektorer som er ortogonale på alle vektorer i $U$ (bemærk at vi ikke benytter at $U$ er lukket). For den omvendte inklusion, lad $u \in (U^\perp)^\perp$ og skriv $u = u' + u''$ med $u' \in U$ og $u'' \in U^\perp$ (jf. Korollar~9.3.5). Så er $\inner{u}{u''} = 0$, hvilket medfører at $u'' = 0$, så $u = u' \in U$.

    Den anden påstand følger hvis vi kan vise at $V^\perp \subseteq \overline{V}^\perp$ (den omvendte inklusion er oplagt). Dette følger af kontinuitet af indre produkter sammen med Opgave~9.3(c).
\end{opgave}


\begin{opgavebreak}[9.5]
\begin{solutionsec}
    \item Alle funktioner er $\calL^2$ (jf. Eksempel~5.2.13). Vi har
    %
    \begin{equation*}
        \inner{f}{g}
            = \sum_{j=1}^n f(j) \overline{g(j)},
    \end{equation*}
    %
    hvilket blot er det sædvanlige indre produkt af (søjle/række)vektorerne $(f(1), \ldots, f(n))$ og $(g(1), \ldots, g(n))$. Altså er afbildningen $\calL^2(\tau_n) \to \complex^n$ givet ved $f \mapsto (f(1), \ldots, f(n))$ en lineær isometri. Dette giver fuldstændigheden af $\complex^n$. Bemærk til sidst at $\emptyset$ er den eneste $\tau_n$-nulmængde.

    \item Bemærk at $f \in \calL^2(\tau_\naturals)$ hvis og kun hvis
    %
    \begin{equation*}
        \sum_{j=1}^\infty \abs{f(j)}^2
            < \infty.
    \end{equation*}
    %
    Et element i $\calL^2(\tau_\naturals)$ er en følge, og dette rum betegnes også $\ell^2(\naturals)$ eller blot $\ell^2$.
\end{solutionsec}
\end{opgavebreak}


\begin{opgavebreak}[9.8]
\begin{solutionsec}
    \item Da $(e_n)$ er en ortonormalbasis, er f.eks. $x = \sum_{n=1}^\infty \inner{x}{e_n} e_n$ ved Korollar~9.4.9. Lad $x' = \sum_{n=1}^\infty \abs{\inner{x}{e_n}} e_n$ (dette giver mening ved Sætning~9.2.3(iii)), og definer $y'$ tilsvarende. Så er $\norm{x} = \norm{x'}$, og kontinuiteten af det indre produkt sammen med Cauchy-Schwarz' ulighed giver at
    %
    \begin{equation*}
        \sum_{n=1}^\infty \abs{\inner{x}{e_n} \inner{y}{e_n}}
            = \inner{x'}{y'}
            \leq \norm{x'} \, \norm{y'}
            = \norm{x} \, \norm{y}.
    \end{equation*}

    \item Følger igen af kontinuiteten af det indre produkt. Sæt $x = y$ for at opnå Parsevals ligning. (Dette er strengt taget ikke en generalisering, da vi i denne opgave antager at $\calH$ er \emph{separabelt}. Men pga. Bemærkning~9.4.5 er det tilstrækkeligt at se på et tælleligt ortonormalsystem svarende til indeksmængden $I_x$.)
\end{solutionsec}
\end{opgavebreak}


\begin{opgave}[8.3]
    Dette er en konsekvens af Opgave~5.24 i lyset af Opgave~8.1.
\end{opgave}

\end{document}