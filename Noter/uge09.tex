\newcommand{\doctitle}{Uge 9}
\newcommand{\docauthor}{Danny Nygård Hansen}

\documentclass[a4paper, 11pt, article, danish, oneside]{memoir}
\usepackage[utf8]{inputenc}
\usepackage[T1]{fontenc}
\usepackage[danish]{babel}
\usepackage[autostyle, danish=guillemets]{csquotes}

\usepackage[final]{microtype}
\frenchspacing
\raggedbottom

\usepackage{mathtools}
\usepackage{amssymb}
\usepackage[largesmallcaps]{kpfonts}
\linespread{1.06}
\DeclareMathAlphabet\mathfrak{U}{euf}{m}{n}
\SetMathAlphabet\mathfrak{bold}{U}{euf}{b}{n}
\usepackage{inconsolata}

\usepackage{hyperref}
\hypersetup{%
	pdftitle=\doctitle,
	pdfauthor={\docauthor},
    hidelinks,
}

\usepackage{enumitem}
\setenumerate[0]{label=\normalfont(\arabic*)}
\setlist{
	listparindent=\parindent,
	parsep=0pt,
}
\usepackage{array}

\title{\doctitle}
\author{\docauthor}

\newcommand{\overbar}[3]{\mkern #1mu\overline{\mkern-#1mu#3\mkern-#2mu}\mkern #2mu}
\newcommand{\naturals}{\mathbb{N}}
\newcommand{\ints}{\mathbb{Z}}
\newcommand{\rationals}{\mathbb{Q}}
\newcommand{\reals}{\mathbb{R}}
\newcommand{\extreals}{\overbar{1.5}{1.5}{\reals}}
\newcommand{\complex}{\mathbb{C}}


\usepackage{pgffor}

\newcommand{\rvar}[1]{\mathsf{#1}}

\foreach \x in {A,...,Z}{%
    \expandafter\xdef\csname cal\x\endcsname{\noexpand\mathcal{\x}}
    \expandafter\xdef\csname frak\x\endcsname{\noexpand\mathfrak{\x}}
    \expandafter\xdef\csname rand\x\endcsname{\noexpand\rvar{\x}}
}


\usepackage{etoolbox}
\newcommand{\blank}{\mathrel{\;\cdot\;}}
\newcommand{\blankifempty}[1]{\ifstrempty{#1}{\blank}{#1}}
\DeclarePairedDelimiter{\auxdelimlvert}{\lvert}{\rvert}
\DeclarePairedDelimiter{\auxdelimlVert}{\lVert}{\rVert}
\DeclarePairedDelimiterX{\auxdelimanglescomma}[2]{\langle}{\rangle}{#1,#2}
\newcommand{\abs}[2][]{\auxdelimlvert[#1]{\blankifempty{#2}}}
\newcommand{\norm}[1]{\auxdelimlVert{\blankifempty{#1}}}
\newcommand{\inner}[2]{\auxdelimanglescomma{\blankifempty{#1}}{\blankifempty{#2}}}


\DeclarePairedDelimiter{\auxdelimparen}{(}{)}
\DeclarePairedDelimiterX{\auxdelimparencomma}[2]{(}{)}{#1,#2}
\DeclarePairedDelimiter{\auxdelimbracket}{[}{]}
\DeclarePairedDelimiterX{\auxdelimbracketcomma}[2]{[}{]}{#1,#2}
\newcommand{\powerset}[2][]{\calP\auxdelimparen[#1]{#2}}
\newcommand{\borel}[2][]{\calB\auxdelimparen[#1]{#2}}
\newcommand{\meas}[2][]{\calM\auxdelimparen[#1]{#2}}
\newcommand{\measC}[2][]{\calM_\complex\auxdelimparen[#1]{#2}}
\newcommand{\measpos}[2][]{\meas[#1]{#2}^+}
\newcommand{\measbound}[2][]{\calM_b\auxdelimparen[#1]{#2}}
\newcommand{\measboundpos}[2][]{\measbound[#1]{#2}^+}


\newcommand{\extmeas}[2][]{\overbar{4.5}{0.5}{\calM}\auxdelimparen[#1]{#2}}
\newcommand{\extmeaspos}[2][]{\extmeas[#1]{#2}^+}
\newcommand{\simplemeas}[2][]{\calS\!\calM\auxdelimparen[#1]{#2}}
\newcommand{\simplemeaspos}[2][]{\simplemeas[#1]{#2}^+}
\newcommand{\sigmaalg}[2][]{\sigma\auxdelimparen[#1]{#2}}
\newcommand{\deltasys}[2][]{\delta\auxdelimparen[#1]{#2}}

\newcommand{\expval}[2][]{\mathbb{E}\auxdelimbracket[#1]{#2}}
\newcommand{\var}[2][]{\operatorname{Var}\auxdelimbracket[#1]{#2}}
\newcommand{\cov}[3][]{\operatorname{Cov}\auxdelimbracketcomma[#1]{#2}{#3}}


\renewcommand{\Re}{\operatorname{Re}}
\renewcommand{\Im}{\operatorname{Im}}
\newcommand{\conj}[1]{\overline{#1}}
\newcommand{\dif}{\mathop{}\!\mathrm{d}}
\DeclareMathOperator{\id}{id}
\newcommand{\indicator}[1]{\mathbf{1}_{#1}}

% Lattice operations
\newcommand{\meet}{\land}
\newcommand{\join}{\lor}

\DeclareMathOperator*{\smallbigvee}{\textstyle\bigvee}
\DeclareMathOperator*{\bigjoin}{\mathchoice
    {\smallbigvee}%
    {\bigvee}%
    {\bigvee}%
    {\bigvee}%
}
\DeclareMathOperator*{\smallbigwedge}{\textstyle\bigwedge}
\DeclareMathOperator*{\bigmeet}{\mathchoice
    {\smallbigwedge}%
    {\bigwedge}%
    {\bigwedge}%
    {\bigwedge}%
}



\newcommand*\union\cup
\newcommand*\intersect\cap

\DeclareMathOperator*{\smallbigcup}{\textstyle\bigcup}
\DeclareMathOperator*{\bigunion}{\mathchoice
    {\smallbigcup}%
    {\bigcup}%
    {\bigcup}%
    {\bigcup}%
}
\DeclareMathOperator*{\smallbigcap}{\textstyle\bigcap}
\DeclareMathOperator*{\bigintersect}{\mathchoice
    {\smallbigcap}%
    {\bigcap}%
    {\bigcap}%
    {\bigcap}%
}


\DeclarePairedDelimiterX{\set}[2]{\lbrace}{\rbrace}{#1\;\delimsize\vert\;#2}

\newcommand{\defeq}{\coloneqq}
\newcommand{\eqdef}{\eqqcolon}
\renewcommand{\phi}{\varphi}
\newcommand{\iu}{\mathrm{i}\mkern1mu}
\DeclareMathOperator{\e}{\mathrm{e}}

\newcommand{\ball}[3][]{%
    \ifstrempty{#1}%
        {%
            b\auxdelimparencomma{#2}{#3}%
        }{%
            b_{#1}\auxdelimparencomma{#2}{#3}%
        }%
}

\newcommand{\converges}[1]{\xrightarrow[#1]{}}
\DeclareMathOperator{\supp}{supp}
\let\oldvec\vec
\renewcommand{\vec}[1]{\underline{#1}}
\newcommand{\Tr}[1][]{%
    \ifstrempty{#1}%
        {%
            \operatorname{Tr}%
        }{%
            \operatorname{Tr}_{#1}%
        }%
}


\usepackage{listofitems}
\setsepchar{,}

\makeatletter
\newcommand{\mat@dims}[1]{%
    \readlist*\@dims{#1}%
    \ifnum \@dimslen=1
        \def\@dimsout{\@dims[1]}%
    \else
        \def\@dimsout{\@dims[1], \@dims[2]}%
    \fi
    \@dimsout
}


\newcommand{\matgroup}[3]{\mathrm{#1}_{#2}(#3)}
\newcommand{\matGL}[2]{\matgroup{GL}{#1}{#2}}
\newcommand{\trans}{^{\top}}
\newcommand{\mat}[2]{M_{\mat@dims{#1}}(#2)}

\makeatother

\DeclareMathOperator{\Span}{span}
\DeclareMathOperator{\clSpan}{\overbar{0.5}{1.5}{span}}

\newcommand\inv{^{-1}}
\newcommand{\preim}[2][]{^{-1}\auxdelimparen[#1]{#2}}

\newcommand{\dsupp}[2][]{\mathrm{Sp}_d\auxdelimparen[#1]{#2}}

\usepackage[amsmath,thmmarks,hyperref]{ntheorem}
\usepackage{bbding}

\newcommand{\theorembullet}{{\footnotesize\textbullet}}
\newcommand{\pencilsymbol}{\raisebox{-2pt}{\normalfont\PencilLeft}}
\makeatletter
\newtheoremstyle{changedotcustomnumber}%
    {}%
    {\item[\hskip\labelsep \theorem@headerfont ##3~~\theorembullet~~##1\theorem@separator]}
\newtheoremstyle{changedotbreakcustomnumber}%
    {}%
    {\item[\rlap{\vbox{\hbox{\hskip\labelsep \theorem@headerfont
            ##3~~\theorembullet~~##1\theorem@separator}\hbox{\strut}}}]}
\makeatother

\theorembodyfont{\normalfont}
\theoremseparator{~~}
\theoremsymbol{\ensuremath{\blacksquare}}
\theoremstyle{changedotcustomnumber}
\newtheorem{opgave}{\pencilsymbol}
\theoremstyle{changedotbreakcustomnumber}
\newtheorem{opgavebreak}{\pencilsymbol}

\newlist{solutionsec}{enumerate}{1}
\setlist[solutionsec]{leftmargin=0pt, parsep=0pt, listparindent=\parindent, label=(\alph*), labelsep=0pt, labelwidth=20pt, itemindent=20pt, align=left, itemsep=.5\baselineskip}


\begin{document}

\maketitle

% • 7.5 (Betragt målrummet
% ({1,...,n},P({1,...,n}),µ) for et passende valgt sandsynlighedsmål µ. Husk også
% Opgave 5.4).
% • 7.6
% • 7.3
% • 7.4
% • 7.11

\begin{opgave}[7.5]
    Betragt sandsynlighedsmålet $\mu$ på $(\naturals, \powerset{\naturals})$ givet ved
    %
    \begin{equation*}
        \mu
            = \frac{\sum_{i=1}^n \ell_i \delta_i}{\sum_{i=1}^n \ell_i},
    \end{equation*}
    %
    og lad $f(i) = x_i$ for $1 \leq i \leq n$ og $f(i) = 0$ ellers. Da er $f \in I$ (næsten) sikkert, så det ønskede følger af Jensens ulighed og Opgave~5.4.
\end{opgave}


\begin{opgavebreak}[7.6]
\begin{solutionsec}
    \item For at se at $\calL^0(\mu)$ er et vektorrum, bemærk at
    %
    \begin{equation*}
        \{ \abs{\alpha f} \geq t \}
            = \{ \abs{f} \geq \abs{\alpha}\inv t \}, 
    \end{equation*}
    %
    og at
    %
    \begin{equation*}
        \{ \abs{f + g} \geq t \}
            \subseteq \{ \abs{f} \geq \tfrac{t}{2} \} \union \{ \abs{g} \geq \tfrac{t}{2} \}.
    \end{equation*}
    %
    At $\calL^\infty(\mu)$ er et vektorrum, følger nærmest direkte af definitionen.

    \item Hvis $f \in \calL^\infty(\mu)$ og $\abs{f} \leq R$ $\mu$-n.o., så er
    %
    \begin{equation*}
        \int_X \abs{f}^p \dif\mu
            \leq R^p \mu(X)
            < \infty.
    \end{equation*}
    %
    Det er desuden oplagt at $\calL^\infty(\mu) \subseteq \calL^0(\mu)$ hvis $\mu$ er endeligt.
    
    Vi har desuden $\calL^p(\mu) \subseteq \calL^0(\mu)$ for $p \in (0,\infty)$ (jf. Bemærkning~7.3.3(3)), så i alt er $\calL^q(\mu) \subseteq \calL^p(\mu)$ hvis $p \leq q$ for alle $p,q \in [0,\infty]$, såfremt $\mu$ er et endeligt mål (jf. også Sætning~7.3.2(ii)). Inklusionen $\calL^p(\mu) \subseteq \calL^0(\mu)$ for $p \in (0,\infty]$ gælder for alle $\mu$, men derudover gælder ingen generelle inklusioner mellem $\calL^p$-rum (se f.eks. Opgave~7.10).
    
    Bemærk også at f.eks. $x \mapsto \e^{1/x}$ er $\calL^0$ (på det \emph{endelige} interval $(0,1]$) men ikke $\calL^p$ for noget $p > 0$. Derudover er ikke alle målelige funktioner $\calL^0$, f.eks. $x \mapsto x$ på $\reals$. Men hvis $\mu$ er endeligt, da er enhver målelig funktion $f$ i $\calL^0(\mu)$: For da er $\bigintersect_{n \in \naturals} \{\abs{f} \geq n\} = \emptyset$, så $\lim_{n \to \infty} \mu(\{\abs{f} \geq n\}) = 0$ pr. kontinuitet.
    
    \item Hvis f.eks. $f,g \in \calL^\infty(\mu)$ og $\abs{f} \leq R_f$ og $\abs{g} \leq R_g$ $\mu$-n.o., da er $\abs{f+g} \leq R_f + R_g$ $\mu$-n.o., så
    %
    \begin{equation*}
        \norm{f + g}_\infty
            \leq R_f + R_g.
    \end{equation*}
    %
    Da $R_f$ og $R_g$ var virkårligt valgt, følger det at $\norm{f+g}_\infty \leq \norm{f}_\infty + \norm{g}_\infty$.
\end{solutionsec}
\end{opgavebreak}


\begin{opgave}[7.3]
    Gentag beviset for Markovs ulighed, men benyt at $1 \leq \psi(\abs{f}) / \psi(t)$.
\end{opgave}


\begin{opgavebreak}[7.4]
\begin{solutionsec}
    \item For den første ulighed, lad $\phi(t) = \sqrt{1 + t^2}$. Denne er kontinuert på $(0,1]$, diffentiabel på $(0,1)$, og $\phi''(t) = (1+t^2)^{-3/2} > 0$, så $\phi$ er konveks pr. Korollar~7.1.3. For den anden ulighed, bemærk at $\sqrt{a+b} \leq \sqrt{a} + \sqrt{b}$ for alle $a,b \in [0,\infty)$.

    \item Vi beskriver først hvordan man formaliserer længden af en kurve. Hvis $I \subseteq \reals$ er et interval, kaldes en kontinuert funktion $f \colon I \to \reals^d$ for en \emph{sti} i $\reals^d$, og billedet $f(I)$ af $f$ kaldes en \emph{kurve}. Antag at $I = [a,b]$. En \emph{partition} af $[a,b]$ er en mængde $P = \{t_0, \ldots, t_n\}$ af tal så
    %
    \begin{equation*}
        a
            = t_0
            < \cdots
            < t_n
            = b.
    \end{equation*}
    %
    Lad $\calP[a,b]$ betegne mængden af partitioner af $[a,b]$. Vi indfører da tallet
    %
    \begin{equation*}
        L_f(P)
            = \sum_{i=1}^n \norm{f(t_i) - f(t_{i-1})}.
    \end{equation*}
    %
    Geometrisk er $L_f(P)$ længden af det indskrevne polygon med hjørner i punkterne $f(t_0), \ldots, f(t_n)$. Vi siger at $f$ er \emph{rektificerbar} hvis tallet
    %
    \begin{equation*}
        L_f(a,b)
            \defeq \sup \set{L_f(P)}{P \in \calP[a,b]}
    \end{equation*}
    %
    er endeligt. I så fald kaldes $L_f(a,b)$ for \emph{længden} af $f$. Man kan vise at hvis $f$ er kontinuert differentiabel, så er $f$ rektificerbar og
    %
    \begin{equation*}
        L_f(a,b)
            = \int_a^b \norm{f'} \dif\lambda.
    \end{equation*}
    %
    Betragt nu tilfældet $d = 2$ og skriv $f = (f_1, f_2)$. Hvis $f_1(t) = t$, da er
    %
    \begin{equation*}
        \norm{f'(t)}^2
            = f_1'(t)^2 + f_2'(t)^2
            = 1 + f_2'(t)^2,
    \end{equation*}
    %
    så
    %
    \begin{equation*}
        L_f(a,b)
            = \int_a^b \sqrt{1 + (f_2')^2} \dif\lambda.
    \end{equation*}
    %
    Hvis $f$ er voksende, da er $f'$ ikke-negativ. Den geometriske fortolkning af opgaven er da at længden af grafen for $f$ er mindst længden af diagonal mellem $(0,0)$ og $(1,1)$, og højst den samlede længde af linjestykkerne fra $(0,0)$ til $(1,0)$ og fra $(1,0)$ til $(1,1)$.
\end{solutionsec}
\end{opgavebreak}


\begin{opgavebreak}[7.11]
\begin{solutionsec}
    \item Antag at $0 \leq \alpha \leq \beta$. Hvis $\alpha = \beta$, er
    %
    \begin{equation*}
        (\alpha + \beta)^p - \beta^p
            = 2^p \alpha^p - \alpha^p
            \leq 2 \alpha^p - \alpha^p
            \leq \alpha^p.
    \end{equation*}
    %
    Antag i stedet at $\alpha \leq \beta$. Funktionen $t \mapsto t^p$ er differentiabel på $(\beta, \alpha+\beta)$ og kontinuert på $[\beta, \alpha+\beta]$, så middelværdisætningen giver et $\xi \in (\beta, \alpha+\beta)$ så $(\alpha+\beta)^p - \beta^p = p\xi^{p-1} \alpha \leq \alpha^p$.

    Sæt $\alpha = \abs{x}$ og $\beta = \abs{y}$.

    \item Trekantsuligheden følger af del (a).
\end{solutionsec}
\end{opgavebreak}


\end{document}