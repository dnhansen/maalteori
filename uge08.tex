\newcommand{\doctitle}{Uge 8}
\newcommand{\docauthor}{Danny Nygård Hansen}

\documentclass[a4paper, 11pt, article, danish, oneside]{memoir}
\usepackage[utf8]{inputenc}
\usepackage[T1]{fontenc}
\usepackage[danish]{babel}
\usepackage[autostyle, danish=guillemets]{csquotes}

\usepackage[final]{microtype}
\frenchspacing
\raggedbottom

\usepackage{mathtools}
\usepackage{amssymb}
\usepackage[largesmallcaps]{kpfonts}
\linespread{1.06}
\DeclareMathAlphabet\mathfrak{U}{euf}{m}{n}
\SetMathAlphabet\mathfrak{bold}{U}{euf}{b}{n}
\usepackage{inconsolata}

\usepackage{hyperref}
\hypersetup{%
	pdftitle=\doctitle,
	pdfauthor={\docauthor},
    hidelinks,
}

\usepackage{enumitem}
\setenumerate[0]{label=\normalfont(\arabic*)}
\setlist{
	listparindent=\parindent,
	parsep=0pt,
}
\usepackage{array}

\title{\doctitle}
\author{\docauthor}

\newcommand{\overbar}[3]{\mkern #1mu\overline{\mkern-#1mu#3\mkern-#2mu}\mkern #2mu}
\newcommand{\naturals}{\mathbb{N}}
\newcommand{\ints}{\mathbb{Z}}
\newcommand{\rationals}{\mathbb{Q}}
\newcommand{\reals}{\mathbb{R}}
\newcommand{\extreals}{\overbar{1.5}{1.5}{\reals}}
\newcommand{\complex}{\mathbb{C}}


\usepackage{pgffor}

\newcommand{\rvar}[1]{\mathsf{#1}}

\foreach \x in {A,...,Z}{%
    \expandafter\xdef\csname cal\x\endcsname{\noexpand\mathcal{\x}}
    \expandafter\xdef\csname frak\x\endcsname{\noexpand\mathfrak{\x}}
    \expandafter\xdef\csname rand\x\endcsname{\noexpand\rvar{\x}}
}


\usepackage{etoolbox}
\newcommand{\blank}{\mathrel{\;\cdot\;}}
\newcommand{\blankifempty}[1]{\ifstrempty{#1}{\blank}{#1}}
\DeclarePairedDelimiter{\auxdelimlvert}{\lvert}{\rvert}
\DeclarePairedDelimiter{\auxdelimlVert}{\lVert}{\rVert}
\DeclarePairedDelimiterX{\auxdelimanglescomma}[2]{\langle}{\rangle}{#1,#2}
\newcommand{\abs}[2][]{\auxdelimlvert[#1]{\blankifempty{#2}}}
\newcommand{\norm}[1]{\auxdelimlVert{\blankifempty{#1}}}
\newcommand{\inner}[2]{\auxdelimanglescomma{\blankifempty{#1}}{\blankifempty{#2}}}


\DeclarePairedDelimiter{\auxdelimparen}{(}{)}
\DeclarePairedDelimiterX{\auxdelimparencomma}[2]{(}{)}{#1,#2}
\DeclarePairedDelimiter{\auxdelimbracket}{[}{]}
\DeclarePairedDelimiterX{\auxdelimbracketcomma}[2]{[}{]}{#1,#2}
\newcommand{\powerset}[2][]{\calP\auxdelimparen[#1]{#2}}
\newcommand{\borel}[2][]{\calB\auxdelimparen[#1]{#2}}
\newcommand{\meas}[2][]{\calM\auxdelimparen[#1]{#2}}
\newcommand{\measC}[2][]{\calM_\complex\auxdelimparen[#1]{#2}}
\newcommand{\measpos}[2][]{\meas[#1]{#2}^+}
\newcommand{\measbound}[2][]{\calM_b\auxdelimparen[#1]{#2}}
\newcommand{\measboundpos}[2][]{\measbound[#1]{#2}^+}


\newcommand{\extmeas}[2][]{\overbar{4.5}{0.5}{\calM}\auxdelimparen[#1]{#2}}
\newcommand{\extmeaspos}[2][]{\extmeas[#1]{#2}^+}
\newcommand{\simplemeas}[2][]{\calS\!\calM\auxdelimparen[#1]{#2}}
\newcommand{\simplemeaspos}[2][]{\simplemeas[#1]{#2}^+}
\newcommand{\sigmaalg}[2][]{\sigma\auxdelimparen[#1]{#2}}
\newcommand{\deltasys}[2][]{\delta\auxdelimparen[#1]{#2}}

\newcommand{\expval}[2][]{\mathbb{E}\auxdelimbracket[#1]{#2}}
\newcommand{\var}[2][]{\operatorname{Var}\auxdelimbracket[#1]{#2}}
\newcommand{\cov}[3][]{\operatorname{Cov}\auxdelimbracketcomma[#1]{#2}{#3}}


\renewcommand{\Re}{\operatorname{Re}}
\renewcommand{\Im}{\operatorname{Im}}
\newcommand{\conj}[1]{\overline{#1}}
\newcommand{\dif}{\mathop{}\!\mathrm{d}}
\DeclareMathOperator{\id}{id}
\newcommand{\indicator}[1]{\mathbf{1}_{#1}}

% Lattice operations
\newcommand{\meet}{\land}
\newcommand{\join}{\lor}

\DeclareMathOperator*{\smallbigvee}{\textstyle\bigvee}
\DeclareMathOperator*{\bigjoin}{\mathchoice
    {\smallbigvee}%
    {\bigvee}%
    {\bigvee}%
    {\bigvee}%
}
\DeclareMathOperator*{\smallbigwedge}{\textstyle\bigwedge}
\DeclareMathOperator*{\bigmeet}{\mathchoice
    {\smallbigwedge}%
    {\bigwedge}%
    {\bigwedge}%
    {\bigwedge}%
}



\newcommand*\union\cup
\newcommand*\intersect\cap

\DeclareMathOperator*{\smallbigcup}{\textstyle\bigcup}
\DeclareMathOperator*{\bigunion}{\mathchoice
    {\smallbigcup}%
    {\bigcup}%
    {\bigcup}%
    {\bigcup}%
}
\DeclareMathOperator*{\smallbigcap}{\textstyle\bigcap}
\DeclareMathOperator*{\bigintersect}{\mathchoice
    {\smallbigcap}%
    {\bigcap}%
    {\bigcap}%
    {\bigcap}%
}


\DeclarePairedDelimiterX{\set}[2]{\lbrace}{\rbrace}{#1\;\delimsize\vert\;#2}

\newcommand{\defeq}{\coloneqq}
\newcommand{\eqdef}{\eqqcolon}
\renewcommand{\phi}{\varphi}
\newcommand{\iu}{\mathrm{i}\mkern1mu}
\DeclareMathOperator{\e}{\mathrm{e}}

\newcommand{\ball}[3][]{%
    \ifstrempty{#1}%
        {%
            b\auxdelimparencomma{#2}{#3}%
        }{%
            b_{#1}\auxdelimparencomma{#2}{#3}%
        }%
}

\newcommand{\converges}[1]{\xrightarrow[#1]{}}
\DeclareMathOperator{\supp}{supp}
\let\oldvec\vec
\renewcommand{\vec}[1]{\underline{#1}}
\newcommand{\Tr}[1][]{%
    \ifstrempty{#1}%
        {%
            \operatorname{Tr}%
        }{%
            \operatorname{Tr}_{#1}%
        }%
}


\usepackage{listofitems}
\setsepchar{,}

\makeatletter
\newcommand{\mat@dims}[1]{%
    \readlist*\@dims{#1}%
    \ifnum \@dimslen=1
        \def\@dimsout{\@dims[1]}%
    \else
        \def\@dimsout{\@dims[1], \@dims[2]}%
    \fi
    \@dimsout
}


\newcommand{\matgroup}[3]{\mathrm{#1}_{#2}(#3)}
\newcommand{\matGL}[2]{\matgroup{GL}{#1}{#2}}
\newcommand{\trans}{^{\top}}
\newcommand{\mat}[2]{M_{\mat@dims{#1}}(#2)}

\makeatother

\DeclareMathOperator{\Span}{span}
\DeclareMathOperator{\clSpan}{\overbar{0.5}{1.5}{span}}

\newcommand\inv{^{\langle-1\rangle}}
\newcommand{\preim}[2][]{^{-1}\auxdelimparen[#1]{#2}}

\newcommand{\dsupp}[2][]{\mathrm{Sp}_d\auxdelimparen[#1]{#2}}

\usepackage[amsmath,thmmarks,hyperref]{ntheorem}
\usepackage{bbding}

\newcommand{\theorembullet}{{\footnotesize\textbullet}}
\newcommand{\pencilsymbol}{\raisebox{-2pt}{\normalfont\PencilLeft}}
\makeatletter
\newtheoremstyle{changedotcustomnumber}%
    {}%
    {\item[\hskip\labelsep \theorem@headerfont ##3~~\theorembullet~~##1\theorem@separator]}
\newtheoremstyle{changedotbreakcustomnumber}%
    {}%
    {\item[\rlap{\vbox{\hbox{\hskip\labelsep \theorem@headerfont
            ##3~~\theorembullet~~##1\theorem@separator}\hbox{\strut}}}]}
\makeatother

\theorembodyfont{\normalfont}
\theoremseparator{~~}
\theoremsymbol{\ensuremath{\blacksquare}}
\theoremstyle{changedotcustomnumber}
\newtheorem{opgave}{\pencilsymbol}
\theoremstyle{changedotbreakcustomnumber}
\newtheorem{opgavebreak}{\pencilsymbol}

\newlist{solutionsec}{enumerate}{1}
\setlist[solutionsec]{leftmargin=0pt, parsep=0pt, listparindent=\parindent, label=(\alph*), labelsep=0pt, labelwidth=20pt, itemindent=20pt, align=left, itemsep=.5\baselineskip}


\begin{document}

\maketitle


\begin{opgave}[5.22]
    Husk at det er tilstrækkeligt at vise at $F(t_n) \to F(t_0)$ for enhver følge $(t_n)_{n \in \naturals}$ i $I$ der konvergerer mod $t_0$. Benyt da domineret konvergens.
\end{opgave}


\begin{opgavebreak}[5.23]
\begin{solutionsec}
    \item Vi benytter her at f.eks.
    %
    \begin{equation*}
        \int_a^b f(x+c) \,\lambda(\dif x)
            = \int_{a+c}^{b+c} f(x) \,\lambda(\dif x),
    \end{equation*}
    %
    hvilket er velkendt for Riemannintegralet og kan vises for Lebesgueintegralet ved den sædvanlige korrespondence derimellem. Dette følger også fra resultater i §11.2. Vis da at
    %
    \begin{equation*}
        \frac{1}{n} \int_a^b \Delta_n f(x) \,\lambda(\dif x)
            = \int_b^{b+1/n} f(x) \,\lambda(\dif x) - \int_a^{a+1/n} f(x) \,\lambda(\dif x),
    \end{equation*}
    %
    og derefter at f.eks.
    %
    \begin{equation*}
        \abs[\bigg]{ n \int_b^{b+1/n} f(x) \,\lambda(\dif x) - f(b) }
            \leq \epsilon
    \end{equation*}
    %
    for $n$ stor nok (benyt her kontinuiteten af $f$ i $b$).

    \item Bemærk at $\Delta_n f \to f'$ punktvist, så $f'$ er målelig. Lad $\abs{f'}$ være begrænset af $R > 0$ på $[a,b+1]$. Ved middelværdisætningen findes for $x \in [a,b]$ et $s \in (x,x+1/n) \subseteq [a,b+1]$ så $\Delta_n f(x) = f'(s)$, så $\abs{\Delta_n f}$ er begrænset af $R$ på $[a,b]$ for alle $n$. Altså er $R\indicator{[a,b]}$ en integrabel majorant. Det ønskede følger da af domineret konvergens.
\end{solutionsec}
\end{opgavebreak}


\begin{opgavebreak}[5.24]
\begin{solutionsec}
    \item Indsæt definitionen på $F$.
    
    \item Benyt middelværdisætningen på funktionen $t \mapsto f(x,t)$.

    \item Benyt at funktionen $g$ er integrabel, og anvend derefter domineret konvergens (bemærk at $\xi_{n,x} \to t$ for $n \to \infty$).

    \item Dette følger af (c) da følgen $(t_n)$ var arbitrært valgt.

    \item Bemærk at $F$ er veldefineret da $\cos$ er begrænset, så integranden er faktisk integrabel. Der findes ikke umiddelbart en funktion $g$ som dominerer den afledte af integranden for alle $t$, men betragt da $F$ restringeret til et interval $(-R,R)$ og vis at den er differentiabel herpå. Da $R$ er vilkårlig, er $F$ differentiabel overalt.
\end{solutionsec}
\end{opgavebreak}


\begin{opgavebreak}[5.25]
\begin{solutionsec}
    \item Bemærk at at $\cos$ er begrænset og at $\mu$ er endeligt, så kontinuitet følger af domineret konvergens.
    
    \item Overvej først at også $\int_\reals x \,\mu(\dif x) < \infty$ da $\abs{x} \leq x^2 + 1$, og $\mu$ er endeligt. Benyt da Opgave~5.24. (Bemærk at eftersom $t$ kun optræder som argument til enten $\sin$ eller $\cos$, behøver vi ikke benytte samme trick som i Opgave~5.24(e).)
\end{solutionsec}
\end{opgavebreak}


\begin{opgavebreak}[6.9]
\begin{solutionsec}
    \item Bemærk at $\tau_1$ er $\sigma$-endeligt, så $\tau_1 \otimes \tau_1$ giver mening. Tjek da at $\tau_2(A \times B) = \tau_1(A) \tau_1(B)$ og benyt entydighedsdelen af Hovedsætning~6.3.3.

    \item Husk Eksempel~5.2.13. Hvis alle $a_{m,n}$ er ikke-negative, da giver Tonelli at summationsrækkefølgen kan ombyttes. Fubini giver det samme såfremt rækken er absolut konvergent.
\end{solutionsec}
\end{opgavebreak}


\begin{opgave}[6.10]
    Løsningsforslag udeladt.
\end{opgave}


\begin{opgavebreak}[7.2]
\begin{solutionsec}
    \item Opskriv hvad det vil sige at $f$ er konveks, benyt at $\phi$ er voksende, og til sidst at $\phi$ er konveks. (Resultatet gælder ikke generelt hvis $\phi$ ikke er voksende. Lad f.eks. $f(x) = x^2$ og $\phi(t) = -t$.)

    \item Benyt at $a = \ell a + (1-\ell)a$ samt trekantsuligheden.

    \item Beregn $\phi''$ og se at denne er positiv, og benyt derefter Korollar~7.1.3.

    \item Dette følger direkte af ovenstående.
\end{solutionsec}
\end{opgavebreak}

\end{document}