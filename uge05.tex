\newcommand{\doctitle}{Uge 5}
\newcommand{\docauthor}{Danny Nygård Hansen}

\documentclass[a4paper, 11pt, article, danish, oneside]{memoir}
\usepackage[utf8]{inputenc}
\usepackage[T1]{fontenc}
\usepackage[danish]{babel}
\usepackage[autostyle, danish=guillemets]{csquotes}

\usepackage[final]{microtype}
\frenchspacing
\raggedbottom

\usepackage{mathtools}
\usepackage{amssymb}
\usepackage[largesmallcaps]{kpfonts}
\linespread{1.06}
\DeclareMathAlphabet\mathfrak{U}{euf}{m}{n}
\SetMathAlphabet\mathfrak{bold}{U}{euf}{b}{n}
\usepackage{inconsolata}

\usepackage{hyperref}
\hypersetup{%
	pdftitle=\doctitle,
	pdfauthor={\docauthor},
    hidelinks,
}

\usepackage{enumitem}
\setenumerate[0]{label=\normalfont(\arabic*)}
\setlist{
	listparindent=\parindent,
	parsep=0pt,
}
\usepackage{array}

\title{\doctitle}
\author{\docauthor}

\newcommand{\overbar}[3]{\mkern #1mu\overline{\mkern-#1mu#3\mkern-#2mu}\mkern #2mu}
\newcommand{\naturals}{\mathbb{N}}
\newcommand{\ints}{\mathbb{Z}}
\newcommand{\rationals}{\mathbb{Q}}
\newcommand{\reals}{\mathbb{R}}
\newcommand{\extreals}{\overbar{1.5}{1.5}{\reals}}
\newcommand{\complex}{\mathbb{C}}


\usepackage{pgffor}

\newcommand{\rvar}[1]{\mathsf{#1}}

\foreach \x in {A,...,Z}{%
    \expandafter\xdef\csname cal\x\endcsname{\noexpand\mathcal{\x}}
    \expandafter\xdef\csname frak\x\endcsname{\noexpand\mathfrak{\x}}
    \expandafter\xdef\csname rand\x\endcsname{\noexpand\rvar{\x}}
}


\usepackage{etoolbox}
\newcommand{\blank}{\mathrel{\;\cdot\;}}
\newcommand{\blankifempty}[1]{\ifstrempty{#1}{\blank}{#1}}
\DeclarePairedDelimiter{\auxdelimlvert}{\lvert}{\rvert}
\DeclarePairedDelimiter{\auxdelimlVert}{\lVert}{\rVert}
\DeclarePairedDelimiterX{\auxdelimanglescomma}[2]{\langle}{\rangle}{#1,#2}
\newcommand{\abs}[1]{\auxdelimlvert{\blankifempty{#1}}}
\newcommand{\norm}[1]{\auxdelimlVert{\blankifempty{#1}}}
\newcommand{\inner}[2]{\auxdelimanglescomma{\blankifempty{#1}}{\blankifempty{#2}}}


\DeclarePairedDelimiter{\auxdelimparen}{(}{)}
\DeclarePairedDelimiterX{\auxdelimparencomma}[2]{(}{)}{#1,#2}
\DeclarePairedDelimiter{\auxdelimbracket}{[}{]}
\DeclarePairedDelimiterX{\auxdelimbracketcomma}[2]{[}{]}{#1,#2}
\newcommand{\powerset}[2][]{\calP\auxdelimparen[#1]{#2}}
\newcommand{\borel}[2][]{\calB\auxdelimparen[#1]{#2}}
\newcommand{\meas}[2][]{\calM\auxdelimparen[#1]{#2}}
\newcommand{\measC}[2][]{\calM_\complex\auxdelimparen[#1]{#2}}
\newcommand{\measpos}[2][]{\meas[#1]{#2}^+}
\newcommand{\measbound}[2][]{\calM_b\auxdelimparen[#1]{#2}}
\newcommand{\measboundpos}[2][]{\measbound[#1]{#2}^+}


\newcommand{\extmeas}[2][]{\overbar{4.5}{0.5}{\calM}\auxdelimparen[#1]{#2}}
\newcommand{\extmeaspos}[2][]{\extmeas[#1]{#2}^+}
\newcommand{\simplemeas}[2][]{\calS\!\calM\auxdelimparen[#1]{#2}}
\newcommand{\simplemeaspos}[2][]{\simplemeas[#1]{#2}^+}
\newcommand{\sigmaalg}[2][]{\sigma\auxdelimparen[#1]{#2}}
\newcommand{\deltasys}[2][]{\delta\auxdelimparen[#1]{#2}}

\newcommand{\expval}[2][]{\mathbb{E}\auxdelimbracket[#1]{#2}}
\newcommand{\var}[2][]{\operatorname{Var}\auxdelimbracket[#1]{#2}}
\newcommand{\cov}[3][]{\operatorname{Cov}\auxdelimbracketcomma[#1]{#2}{#3}}


\renewcommand{\Re}{\operatorname{Re}}
\renewcommand{\Im}{\operatorname{Im}}
\newcommand{\conj}[1]{\overline{#1}}
\newcommand{\dif}{\mathop{}\!\mathrm{d}}
\DeclareMathOperator{\id}{id}
\newcommand{\indicator}[1]{\mathbf{1}_{#1}}

% Lattice operations
\newcommand{\meet}{\land}
\newcommand{\join}{\lor}

\DeclareMathOperator*{\smallbigvee}{\textstyle\bigvee}
\DeclareMathOperator*{\bigjoin}{\mathchoice
    {\smallbigvee}%
    {\bigvee}%
    {\bigvee}%
    {\bigvee}%
}
\DeclareMathOperator*{\smallbigwedge}{\textstyle\bigwedge}
\DeclareMathOperator*{\bigmeet}{\mathchoice
    {\smallbigwedge}%
    {\bigwedge}%
    {\bigwedge}%
    {\bigwedge}%
}



\newcommand*\union\cup
\newcommand*\intersect\cap

\DeclareMathOperator*{\smallbigcup}{\textstyle\bigcup}
\DeclareMathOperator*{\bigunion}{\mathchoice
    {\smallbigcup}%
    {\bigcup}%
    {\bigcup}%
    {\bigcup}%
}
\DeclareMathOperator*{\smallbigcap}{\textstyle\bigcap}
\DeclareMathOperator*{\bigintersect}{\mathchoice
    {\smallbigcap}%
    {\bigcap}%
    {\bigcap}%
    {\bigcap}%
}


\DeclarePairedDelimiterX{\set}[2]{\lbrace}{\rbrace}{#1\;\delimsize\vert\;#2}

\newcommand{\defeq}{\coloneqq}
\newcommand{\eqdef}{\eqqcolon}
\renewcommand{\phi}{\varphi}
\newcommand{\iu}{\mathrm{i}\mkern1mu}
\DeclareMathOperator{\e}{\mathrm{e}}

\newcommand{\ball}[3][]{%
    \ifstrempty{#1}%
        {%
            b\auxdelimparencomma{#2}{#3}%
        }{%
            b_{#1}\auxdelimparencomma{#2}{#3}%
        }%
}

\newcommand{\converges}[1]{\xrightarrow[#1]{}}
\DeclareMathOperator{\supp}{supp}
\let\oldvec\vec
\renewcommand{\vec}[1]{\underline{#1}}
\newcommand{\Tr}[1][]{%
    \ifstrempty{#1}%
        {%
            \operatorname{Tr}%
        }{%
            \operatorname{Tr}_{#1}%
        }%
}


\usepackage{listofitems}
\setsepchar{,}

\makeatletter
\newcommand{\mat@dims}[1]{%
    \readlist*\@dims{#1}%
    \ifnum \@dimslen=1
        \def\@dimsout{\@dims[1]}%
    \else
        \def\@dimsout{\@dims[1], \@dims[2]}%
    \fi
    \@dimsout
}


\newcommand{\matgroup}[3]{\mathrm{#1}_{#2}(#3)}
\newcommand{\matGL}[2]{\matgroup{GL}{#1}{#2}}
\newcommand{\trans}{^{\top}}
\newcommand{\mat}[2]{M_{\mat@dims{#1}}(#2)}

\makeatother

\DeclareMathOperator{\Span}{span}
\DeclareMathOperator{\clSpan}{\overbar{0.5}{1.5}{span}}

\newcommand\inv{^{\langle-1\rangle}}
\newcommand{\preim}[2][]{^{-1}\auxdelimparen[#1]{#2}}

\newcommand{\dsupp}[2][]{\mathrm{Sp}_d\auxdelimparen[#1]{#2}}

\usepackage[amsmath,thmmarks,hyperref]{ntheorem}
\usepackage{bbding}

\newcommand{\theorembullet}{{\footnotesize\textbullet}}
\newcommand{\pencilsymbol}{\raisebox{-2pt}{\normalfont\PencilLeft}}
\makeatletter
\newtheoremstyle{changedotcustomnumber}%
    {}%
    {\item[\hskip\labelsep \theorem@headerfont ##3~~\theorembullet~~##1\theorem@separator]}
\newtheoremstyle{changedotbreakcustomnumber}%
    {}%
    {\item[\rlap{\vbox{\hbox{\hskip\labelsep \theorem@headerfont
            ##3~~\theorembullet~~##1\theorem@separator}\hbox{\strut}}}]}
\makeatother

\theorembodyfont{\normalfont}
\theoremseparator{~~}
\theoremsymbol{\ensuremath{\blacksquare}}
\theoremstyle{changedotcustomnumber}
\newtheorem{opgave}{\pencilsymbol}
\theoremstyle{changedotbreakcustomnumber}
\newtheorem{opgavebreak}{\pencilsymbol}

\newlist{solutionsec}{enumerate}{1}
\setlist[solutionsec]{leftmargin=0pt, parsep=0pt, listparindent=\parindent, label=(\alph*), labelsep=0pt, labelwidth=20pt, itemindent=20pt, align=left, itemsep=.5\baselineskip}


\begin{document}

\maketitle

% • 5.2, 5.5,
% 5.3,
% Hvis der er tid til overs kan man desuden regne
% og/eller Opgave 5.4.

\begin{opgave}[4.8]
    Bemærk at
    %
    \begin{multline*}
        \set{(x,y) \in \reals^2}{x > 0 \text{, og } 0 \leq y \leq \tfrac{1}{x}} \\
            = p_1\preim{(0,\infty)} \intersect p_2\preim{[0,\infty)} \intersect (g-p_2)\preim{[0,\infty)},
    \end{multline*}
    %
    hvor $g(x,y) = \tfrac{1}{x}$ for $x > 0$ og f.eks. $g(x,y) = 0$ (det sidste er ligegyldigt da $x > 0$).
\end{opgave}


\begin{opgave}[4.9]
    Benyt at funktionen $x \mapsto \abs{x}$ er kontinuert. For at modbevise den omvendte implikation, lad $A$ være en ikke-målelig delmængde af $X$ og betragt funktionen $\indicator{A} - \indicator{A^c}$.
\end{opgave}


\begin{opgavebreak}[4.11]
\begin{solutionsec}
    \item For $t \in \reals$ og $\epsilon > 0$, lad $\delta > 0$ afparere $\epsilon$ i $t$ og lad $n > \frac{1}{\delta}$. Hvis $n$ er stor nok, da er $t \in [\tfrac{k-1}{n},\tfrac{k}{n})$ for et $k$ mellem $-n^2+1$ og $n^2$. Det følger at $\tfrac{k}{n} - t < \delta$, så $\abs{f(\tfrac{k}{n}) - f(t)} < \epsilon$.

    \item Bemærk at alle $f_n$ er målelige.

    \item Ligner (a).
\end{solutionsec}
\end{opgavebreak}


\begin{opgavebreak}[4.14]
\begin{solutionsec}
    \item Bemærk at $V \subseteq \extmeaspos{\calE}$ da $\extmeaspos{\calE}$ indeholder de målelige indikatorfunktioner og er lukket under de nævnte operationer. For den modsatte inklusion, benyt som nævnt Sætning~4.5.3.

    \item Samme fremgangsmåde som del (a).
\end{solutionsec}
\end{opgavebreak}


\begin{opgave}[4.15]
    De to identiteter vises på samme vis. Bemærk at
    %
    \begin{equation*}
        \indicator{B} \circ \phi
            = \indicator{\phi\preim{B}}
        \quad \text{og at} \quad
        (\alpha f + \beta g) \circ \phi
            = \alpha (f \circ \phi) + \beta (g \circ \phi).
    \end{equation*}
    %
    Det er relativt besværligt at vise at mængderne på højre side af identiteterne er lukket under voksende grænseovergang. Hvis $(g_n) = (f_n \circ \phi)$ er en voksende følge deri, da er $g = f \circ \phi$, hvor $g = \sup_{n \in \naturals} g_n = \lim_{n \to \infty} g_n$ og $f = \sup_{n \in \naturals} f_n$. Men selvom alle $g_n$ tager værdier i $\reals$, gælder dette ikke nødvendigvis for $\sup_{n \in \naturals} f_n$, så i det andet tilfælde kan vi få problemer. I dette tilfælde kan vi dog antage (jf. Opgave~4.14(b)(III)) at $g < \infty$ overalt. Hvis $f(y) = \infty$ for et $y \in Y$, da kan $y$ altså ikke ligge i $\phi(X)$. Altså er $\phi(X) \subseteq \{f < \infty\} \eqdef B$. Derfor må vi også have $g = f\indicator{B} \circ \phi$, men $f\indicator{B} \in \meas{\calF}$ som ønsket (husk at $f \in \extmeas{\calF}$ ved Sætning~4.3.6, og $B \in \calF$).

    Hvis man ikke har noget imod at antage at $\phi(X) \in \calF$, da kan man blot betragte $f\indicator{\phi(X)}$ i stedet for $f\indicator{B}$, hvilket gør argumentet en smule lettere. Men dette koster altså en ekstra antagelse!
\end{opgave}


\begin{opgavebreak}[5.2]
\begin{solutionsec}
    \item Benyt blot linearitet af integralet.

    \item En grænseværdi af målelige funktioner er målelig. Benyt da monoton konvergens. At mængderne $A_j$ er disjunkte betyder blot at $s$ er endelig, ellers er denne antagelse ikke nødvendig.

    \item Skriv $s$ på formen
    %
    \begin{equation*}
        s
            = \sum_{n=1}^\infty \frac{1}{n^2} \indicator{(n-1,n]}
    \end{equation*}
    %
    og benyt del (b).
\end{solutionsec}
\end{opgavebreak}


\begin{opgave}[5.3]
    Hvis man benytter Hovedsætning~5.2.11, da følger (i1) og (i2) ved blot at indsætte $E_{\delta_a}(f) = f(a)$, og (i3) følger da vi netop betragter den punktvise grænse af $(f_n)$.
\end{opgave}


\begin{opgave}[5.4]
    Som i Eksempel~5.2.13 kan vi skrive
    %
    \begin{equation*}
        f(n)
            = \sum_{k=1}^\infty f(k) \indicator{\{k\}}(n),
    \end{equation*}
    %
    og derefter benytte Sætning~5.2.9 til at slutte at
    %
    \begin{equation*}
        \int f \dif\mu
            = \sum_{k=1}^\infty f(k) \int \indicator{\{k\}} \dif\mu.
    \end{equation*}
    %
    Tilbage er at bemærke at
    %
    \begin{equation*}
        \int \indicator{\{k\}} \dif\mu
            = \mu(\{k\})
            = \alpha_n.
    \end{equation*}

    Man kan også benytte Hovedsætning~5.2.11, men da skal man bruge at det er tilladt at bytte om på grænseværdier -- et resultat som minder om Lemma~A.2.14, men hvor der kun er tale om en enkelt sum.
\end{opgave}


\begin{opgavebreak}[5.5]
\begin{solutionsec}
    \item Benyt Hovedsætning~5.7.3 (eller Sætning~A på ugesedlen) og monoton konvergens.

    \item Som i del (a).
\end{solutionsec}
\end{opgavebreak}

\end{document}