\newcommand{\doctitle}{Uge 6}
\newcommand{\docauthor}{Danny Nygård Hansen}

\documentclass[a4paper, 11pt, article, danish, oneside]{memoir}
\usepackage[utf8]{inputenc}
\usepackage[T1]{fontenc}
\usepackage[danish]{babel}
\usepackage[autostyle, danish=guillemets]{csquotes}

\usepackage[final]{microtype}
\frenchspacing
\raggedbottom

\usepackage{mathtools}
\usepackage{amssymb}
\usepackage[largesmallcaps]{kpfonts}
\linespread{1.06}
\DeclareMathAlphabet\mathfrak{U}{euf}{m}{n}
\SetMathAlphabet\mathfrak{bold}{U}{euf}{b}{n}
\usepackage{inconsolata}

\usepackage{hyperref}
\hypersetup{%
	pdftitle=\doctitle,
	pdfauthor={\docauthor},
    hidelinks,
}

\usepackage{enumitem}
\setenumerate[0]{label=\normalfont(\arabic*)}
\setlist{
	listparindent=\parindent,
	parsep=0pt,
}
\usepackage{array}

\title{\doctitle}
\author{\docauthor}

\newcommand{\overbar}[3]{\mkern #1mu\overline{\mkern-#1mu#3\mkern-#2mu}\mkern #2mu}
\newcommand{\naturals}{\mathbb{N}}
\newcommand{\ints}{\mathbb{Z}}
\newcommand{\rationals}{\mathbb{Q}}
\newcommand{\reals}{\mathbb{R}}
\newcommand{\extreals}{\overbar{1.5}{1.5}{\reals}}
\newcommand{\complex}{\mathbb{C}}


\usepackage{pgffor}

\newcommand{\rvar}[1]{\mathsf{#1}}

\foreach \x in {A,...,Z}{%
    \expandafter\xdef\csname cal\x\endcsname{\noexpand\mathcal{\x}}
    \expandafter\xdef\csname frak\x\endcsname{\noexpand\mathfrak{\x}}
    \expandafter\xdef\csname rand\x\endcsname{\noexpand\rvar{\x}}
}


\usepackage{etoolbox}
\newcommand{\blank}{\mathrel{\;\cdot\;}}
\newcommand{\blankifempty}[1]{\ifstrempty{#1}{\blank}{#1}}
\DeclarePairedDelimiter{\auxdelimlvert}{\lvert}{\rvert}
\DeclarePairedDelimiter{\auxdelimlVert}{\lVert}{\rVert}
\DeclarePairedDelimiterX{\auxdelimanglescomma}[2]{\langle}{\rangle}{#1,#2}
\newcommand{\abs}[1]{\auxdelimlvert{\blankifempty{#1}}}
\newcommand{\norm}[1]{\auxdelimlVert{\blankifempty{#1}}}
\newcommand{\inner}[2]{\auxdelimanglescomma{\blankifempty{#1}}{\blankifempty{#2}}}


\DeclarePairedDelimiter{\auxdelimparen}{(}{)}
\DeclarePairedDelimiterX{\auxdelimparencomma}[2]{(}{)}{#1,#2}
\DeclarePairedDelimiter{\auxdelimbracket}{[}{]}
\DeclarePairedDelimiterX{\auxdelimbracketcomma}[2]{[}{]}{#1,#2}
\newcommand{\powerset}[2][]{\calP\auxdelimparen[#1]{#2}}
\newcommand{\borel}[2][]{\calB\auxdelimparen[#1]{#2}}
\newcommand{\meas}[2][]{\calM\auxdelimparen[#1]{#2}}
\newcommand{\measC}[2][]{\calM_\complex\auxdelimparen[#1]{#2}}
\newcommand{\measpos}[2][]{\meas[#1]{#2}^+}
\newcommand{\measbound}[2][]{\calM_b\auxdelimparen[#1]{#2}}
\newcommand{\measboundpos}[2][]{\measbound[#1]{#2}^+}


\newcommand{\extmeas}[2][]{\overbar{4.5}{0.5}{\calM}\auxdelimparen[#1]{#2}}
\newcommand{\extmeaspos}[2][]{\extmeas[#1]{#2}^+}
\newcommand{\simplemeas}[2][]{\calS\!\calM\auxdelimparen[#1]{#2}}
\newcommand{\simplemeaspos}[2][]{\simplemeas[#1]{#2}^+}
\newcommand{\sigmaalg}[2][]{\sigma\auxdelimparen[#1]{#2}}
\newcommand{\deltasys}[2][]{\delta\auxdelimparen[#1]{#2}}

\newcommand{\expval}[2][]{\mathbb{E}\auxdelimbracket[#1]{#2}}
\newcommand{\var}[2][]{\operatorname{Var}\auxdelimbracket[#1]{#2}}
\newcommand{\cov}[3][]{\operatorname{Cov}\auxdelimbracketcomma[#1]{#2}{#3}}


\renewcommand{\Re}{\operatorname{Re}}
\renewcommand{\Im}{\operatorname{Im}}
\newcommand{\conj}[1]{\overline{#1}}
\newcommand{\dif}{\mathop{}\!\mathrm{d}}
\DeclareMathOperator{\id}{id}
\newcommand{\indicator}[1]{\mathbf{1}_{#1}}

% Lattice operations
\newcommand{\meet}{\land}
\newcommand{\join}{\lor}

\DeclareMathOperator*{\smallbigvee}{\textstyle\bigvee}
\DeclareMathOperator*{\bigjoin}{\mathchoice
    {\smallbigvee}%
    {\bigvee}%
    {\bigvee}%
    {\bigvee}%
}
\DeclareMathOperator*{\smallbigwedge}{\textstyle\bigwedge}
\DeclareMathOperator*{\bigmeet}{\mathchoice
    {\smallbigwedge}%
    {\bigwedge}%
    {\bigwedge}%
    {\bigwedge}%
}



\newcommand*\union\cup
\newcommand*\intersect\cap

\DeclareMathOperator*{\smallbigcup}{\textstyle\bigcup}
\DeclareMathOperator*{\bigunion}{\mathchoice
    {\smallbigcup}%
    {\bigcup}%
    {\bigcup}%
    {\bigcup}%
}
\DeclareMathOperator*{\smallbigcap}{\textstyle\bigcap}
\DeclareMathOperator*{\bigintersect}{\mathchoice
    {\smallbigcap}%
    {\bigcap}%
    {\bigcap}%
    {\bigcap}%
}


\DeclarePairedDelimiterX{\set}[2]{\lbrace}{\rbrace}{#1\;\delimsize\vert\;#2}

\newcommand{\defeq}{\coloneqq}
\newcommand{\eqdef}{\eqqcolon}
\renewcommand{\phi}{\varphi}
\newcommand{\iu}{\mathrm{i}\mkern1mu}
\DeclareMathOperator{\e}{\mathrm{e}}

\newcommand{\ball}[3][]{%
    \ifstrempty{#1}%
        {%
            b\auxdelimparencomma{#2}{#3}%
        }{%
            b_{#1}\auxdelimparencomma{#2}{#3}%
        }%
}

\newcommand{\converges}[1]{\xrightarrow[#1]{}}
\DeclareMathOperator{\supp}{supp}
\let\oldvec\vec
\renewcommand{\vec}[1]{\underline{#1}}
\newcommand{\Tr}[1][]{%
    \ifstrempty{#1}%
        {%
            \operatorname{Tr}%
        }{%
            \operatorname{Tr}_{#1}%
        }%
}


\usepackage{listofitems}
\setsepchar{,}

\makeatletter
\newcommand{\mat@dims}[1]{%
    \readlist*\@dims{#1}%
    \ifnum \@dimslen=1
        \def\@dimsout{\@dims[1]}%
    \else
        \def\@dimsout{\@dims[1], \@dims[2]}%
    \fi
    \@dimsout
}


\newcommand{\matgroup}[3]{\mathrm{#1}_{#2}(#3)}
\newcommand{\matGL}[2]{\matgroup{GL}{#1}{#2}}
\newcommand{\trans}{^{\top}}
\newcommand{\mat}[2]{M_{\mat@dims{#1}}(#2)}

\makeatother

\DeclareMathOperator{\Span}{span}
\DeclareMathOperator{\clSpan}{\overbar{0.5}{1.5}{span}}

\newcommand\inv{^{\langle-1\rangle}}
\newcommand{\preim}[2][]{^{-1}\auxdelimparen[#1]{#2}}

\newcommand{\dsupp}[2][]{\mathrm{Sp}_d\auxdelimparen[#1]{#2}}

\usepackage[amsmath,thmmarks,hyperref]{ntheorem}
\usepackage{bbding}

\newcommand{\theorembullet}{{\footnotesize\textbullet}}
\newcommand{\pencilsymbol}{\raisebox{-2pt}{\normalfont\PencilLeft}}
\makeatletter
\newtheoremstyle{changedotcustomnumber}%
    {}%
    {\item[\hskip\labelsep \theorem@headerfont ##3~~\theorembullet~~##1\theorem@separator]}
\newtheoremstyle{changedotbreakcustomnumber}%
    {}%
    {\item[\rlap{\vbox{\hbox{\hskip\labelsep \theorem@headerfont
            ##3~~\theorembullet~~##1\theorem@separator}\hbox{\strut}}}]}
\makeatother

\theorembodyfont{\normalfont}
\theoremseparator{~~}
\theoremsymbol{\ensuremath{\blacksquare}}
\theoremstyle{changedotcustomnumber}
\newtheorem{opgave}{\pencilsymbol}
\theoremstyle{changedotbreakcustomnumber}
\newtheorem{opgavebreak}{\pencilsymbol}

\newlist{solutionsec}{enumerate}{1}
\setlist[solutionsec]{leftmargin=0pt, parsep=0pt, listparindent=\parindent, label=(\alph*), labelsep=0pt, labelwidth=20pt, itemindent=20pt, align=left, itemsep=.5\baselineskip}


\begin{document}

\maketitle

% • 5.13, 5.11
% • 5.9, 5.10,
% • 5.15, 5.16, 5.17.

\begin{opgavebreak}[5.9]
\begin{solutionsec}
    \item Fra Opgave~1.3(b) ved vi at enhver tællelig delmængde er målelig. Benyt da Opgave~1.14(a).

    \item Dette vises let ved kontraponering, da hvis $\reals \setminus N$ ikke er tæt i $\reals$, da indeholder $N$ et åbent interval. Den omvendte implikation gælder ikke (betragt f.eks. $N = \rationals$).

    \item Mængden $\{f \neq g\}$ er en nulmængde, så $\{f = g\}$ er tæt i $\reals$. Udvid da til dele $\reals$ ved at betragte passende følger i $\{f = g\}$.

    \item Antag at $f$ er en sådan funktion. Der findes da et $\delta > 0$ der afparerer $1$ i punktet $0$. Men intervallet $(-\delta,\delta)$ må både indeholder punkter $x$ hvor $f(x) = 0$ og $f(x) = 1$ (pga. del (a)), hvilket er en modstrid.
\end{solutionsec}
\end{opgavebreak}


\begin{opgave}[5.10]
    Beregn blot integralerne af $f$ og $g$, og bemærk at $(f + g)^+ = g$ og $(f + g)^- = -f$.
\end{opgave}


\begin{opgavebreak}[5.11]
\begin{solutionsec}
    \item Den eneste $\tau$-nulmængde er $\emptyset$.

    \item Fra Eksempel~5.2.13 ved vi at $\int f \dif\tau = \sum_{n=1}^\infty f(n)$ for $f \colon \naturals \to [0,\infty]$. Se da på $f^+$ og $f^-$ og beskriv $\calL(\tau)$ og $\calL^1(\tau)$ i termer af summer.

    \item Benyt del (b). For absolut konvergens, benyt karakterisationen af $\calL^1(\tau)$ fra Sætning~5.4.3.

    \item Split både integralet og summen op i udtryk der hver især afhænger af $f^+$ og $f^-$, benyt Eksempel~5.2.13 igen, og saml til sidst summerne (her skal man være en smule varsom, da man ikke blot kan lægge divergerende summer sammen -- men her er der intet problem da den ene sum divergerer mod \emph{uendelig}).
\end{solutionsec}
\end{opgavebreak}


\begin{opgavebreak}[5.13]
\begin{solutionsec}
    \item Eftersom $f_n(0) = 0$, går følgen mod nulfunktionen. Omvendt er integralet konstant lig med $1$. Bemærk at vi ikke kan anvende monoton konvergens, da følgen ikke er monoton. Derudover vil enhver majorant for følgen være punktvist mindre end funktionen $g$ givet ved
    %
    \begin{equation*}
        g(x)
            = \sum_{n=1}^\infty n \indicator{\bigl(\tfrac{1}{n+1},\tfrac{1}{n}\bigr]},
    \end{equation*}
    %
    men
    %
    \begin{equation*}
        \int g \dif\lambda
            = \sum_{n=1}^\infty n \Bigl( \frac{1}{n} - \frac{1}{n+1} \Bigr)
            = \sum_{n=1}^\infty \frac{1}{n+1},
    \end{equation*}
    %
    og denne række divergerer mod $\infty$. Altså har $(f_n)$ ingen integrabel majorant, og vi kan derfor heller ikke anvende domineret konvergens. (Eftersom domineret konvergens ikke kan anvendes her, så ved vi allerede at $(f_n)$ ikke har en integrabel majorant. Men ovenstående er et mere direkte argument for dette.)

    \item For $x \not\in [0,1]$ er $g_n(x)$ konstant lig med $0$, og $g_n(1)$ er konstant lig med $g(1)$. For $x \in [0,1)$ vil $x^n \to 0$ for $n \to \infty$,\footnote{Vi husker dette fra \emph{Matematisk analyse 1}. Det følger af Benoullis ulighed og af den arkimediske egenskab ved de reelle tal (dvs., der findes vilkårligt store naturlige tal).} så $g(x^n) \to g(0)$ for $n \to \infty$ ved kontinuitet.

    Eftersom $g$ er kontinuert på $[0,1]$, er den begrænset derpå at et tal $R > 0$. Så er $g_n$ også begrænset af $R$, så funktionen $R \indicator{[0,1]}$ er en integrabel majorant for $(g_n)$. Ved domineret konvergens får vi at
    %
    \begin{equation*}
        \lim_{n \to \infty} \int g_n \dif\lambda
            = \int \lim_{n \to \infty} g_n \dif\lambda
            = \int \bigl( g(0) \indicator{[0,1)} + g(1) \indicator{\{1\}} \bigr) \dif\lambda
            = g(0).
    \end{equation*}
\end{solutionsec}
\end{opgavebreak}


\begin{opgavebreak}[5.15]
\begin{solutionsec}
    \item Dette følger da $g^+ \leq h^+$ og $g^- \leq f^-$.

    \item At $g \in \calL^1(\mu)$ følger fra del (a). Ulighederne følger ved at tage integralet på alle sider af uligheden $a \leq g \leq b$.

    \item Bemærk at
    %
    \begin{equation*}
        \int \abs{fh} \dif\mu
            \leq \int K \abs{h} \dif\mu
            = K \int \abs{h} \dif\mu
            < \infty.
    \end{equation*}

    \item Lad $\mu = \lambda^r_{(0,1]}$, og lad $f(x) = h(x) = 1/\sqrt{x}$. Da er $f$ (og $h$) integrabel (den har stamfunktion $x \mapsto 2\sqrt{x}$), men $fh$ er ikke integrabel.
\end{solutionsec}
\end{opgavebreak}


\begin{opgave}[5.16]
    Bemærk først at hvis $y > 0$, så er
    %
    \begin{equation*}
        0
            \leq \biggl( \frac{1}{\sqrt{y}} - \sqrt{y} \biggr)^2
            = \frac{1}{y} + y - 2,
    \end{equation*}
    %
    hvilket medfører at $2 \leq \frac{1}{y} + y$. Lader vi $y = nx$, får vi at
    %
    \begin{equation*}
        2
            \leq \frac{1}{nx} + nx
            = \frac{1 + n^2 x^2}{nx},
    \end{equation*}
    %
    og ved at gange med $\sqrt{x}$,
    %
    \begin{equation*}
        2\sqrt{x}
            \leq \frac{1 + n^2 x^2}{n\sqrt{x}}.
    \end{equation*}
    %
    Den ønskede ulighed følger ved at tage den reciprokke på begge sider (og vende uligheden).

    Ved Opgave~5.14 er funktionen $x \mapsto 1/(2\sqrt{x}) \cdot \indicator{(0,1]}$ en integrabel majorant for følgen af integrander, så det ønskede følger af domineret konvergens, idet integranderne går mod nulfunktionen for $n \to \infty$. (Bemærk desuden at faktoren $\frac{1}{2}$ er helt ligegyldig for det sidste argument.)
\end{opgave}


\begin{opgavebreak}[5.17]
\begin{solutionsec}
    \item Da $\ln$ har stamfunktion $1/x$ på $(0,\infty)$, følger dette ved at omskrive til et Riemannintegral vha. monoton konvergens, og derefter benytte analysens fundamentalsætning.

    \item Bemærk at $f$ er begrænset af $\ln 2$ på $[-2,2]$, og er dermed integrabel. For at beregne integralet, benyt den sædvanlige metode som ovenfor.
\end{solutionsec}
\end{opgavebreak}

\end{document}